\documentclass[10pt]{article}
\usepackage[utf8]{inputenc}
\usepackage[T1]{fontenc}
\usepackage{hyperref}
\hypersetup{colorlinks=true, linkcolor=blue, filecolor=magenta, urlcolor=cyan,}
\urlstyle{same}
\usepackage{amsmath}
\usepackage{amsfonts}
\usepackage{amssymb}
\usepackage[version=4]{mhchem}
\usepackage{stmaryrd}
\usepackage{caption}
\usepackage{multirow}
\usepackage{graphicx}
\usepackage[export]{adjustbox}
\graphicspath{ {./images/} }

\title{PORTFOLIOS OF THE RICH }

\author{Christopher D.Carroll\\
Department of Economics\\
The Johns Hopkins University\\
Baltimore, MD 21218-2685\\
and NBER\\
ccarroll@jhu.edu\\
(410)516-7602 (office)\\
(303)-845-7533 (fax)}
\date{}


%New command to display footnote whose markers will always be hidden
\let\svthefootnote\thefootnote
\newcommand\blfootnotetext[1]{%
  \let\thefootnote\relax\footnote{#1}%
  \addtocounter{footnote}{-1}%
  \let\thefootnote\svthefootnote%
}

%Overriding the \footnotetext command to hide the marker if its value is `0`
\let\svfootnotetext\footnotetext
\renewcommand\footnotetext[2][?]{%
  \if\relax#1\relax%
    \ifnum\value{footnote}=0\blfootnotetext{#2}\else\svfootnotetext{#2}\fi%
  \else%
    \if?#1\ifnum\value{footnote}=0\blfootnotetext{#2}\else\svfootnotetext{#2}\fi%
    \else\svfootnotetext[#1]{#2}\fi%
  \fi
}

\DeclareUnicodeCharacter{2020}{\ifmmode\dagger\else{$\dagger$}\fi}

\begin{document}
\maketitle
\captionsetup{singlelinecheck=false}
Working Paper 7826\\
\href{http://www.nber.org/papers/w7826}{http://www.nber.org/papers/w7826}

\section*{NATIONAL BUREAU OF ECONOMIC RESEARCH \\
 1050 Massachusetts Avenue }
Cambridge, MA 02138\\
August 2000

The original version of this paper was prepared for the conference "Household Portfolios" held at the European University Institute on December 11-12 1999, and this paper will be included in the corresponding conference volume with the same title, edited by Luigi Guiso, Michael Haliassos, and Tullio Jappelli and to be published by MIT Press. I am grateful to Kevin Moore for excellent research assistance, and to Marco Pagano and other participants in the conference for numerous useful suggestion and ideas. An archive of all of the program and data used to produce the table, with extensive instructions, can be found at my website, \href{http://www.econ.jhu.edu/people/ccarroll/carroll.html}{http://www.econ.jhu.edu/people/ccarroll/carroll.html}. The views expressed herein are those of the author and not necessarily those of the National Bureau of Economic Research.\\
© 2000 by Christopher D. Carroll. All rights reserved. Short sections of text, not to exceed two paragraphs, may be quoted without explicit permission provided that full credit, including © notice, is given to the source.



\begin{abstract}
Recent research has shown that 'rich' households save at much higher rates than others (see Carroll (2000); Dynan, Skinner, and Zeldes (1996); Gentry and Hubbard (1998); Huggett (1996); Quadrini (1999)). This paper documents another large difference between the rich and the rest of the population: portfolios of the rich are heavily skewed toward risky assets, particularly investments in their own privately held businesses. The paper explores three possible explanations of these facts. First, perhaps there is exogenous variation in risk tolerance, so that highly risk tolerant house-holds engage in high-risk, high-return activities, and the risk-lovers who are lucky constitute the rich. A second possibility is that capital market imperfections a la Gentry and Hubbard (1998)and Quadrini (1999) require entrepreneurial activities to be largely self-financed, and these same imperfections imply that entreprenurial investment will yield high average returns. The final possibility is that wealth enters households' utility functions directly as a luxury good as in Carroll (2000) (one interpretation is that this reflects the utility of anticipated bequests), implying that risk aversion declines as wealth rises. The paper concludes that the overall pattern of facts suggests both Carrollstyle utility and Gentry/Hubbard-Quadrini style capital market imperfections are important.
\end{abstract}



\section*{1 Introduction}
Ever since the pathbreaking work of Pareto more than a century ago, economists have known that wealth is extremely unevenly distributed. More recently, survey data have revealed that portfolio structures are also very different for households with different levels of wealth. While the portfolios of the rich are complex, the portfolio of financial and real assets of the median household (at least in the U.S.) is rather simple: a checking/savings account plus a home and mortgage, and not much else. ${ }^{1}$ Overwhelmingly, the data tell us that if we wish to understand aggregate portfolio behavior, it is critical to understand the behavior of the richest few percent of households, both because they control the bulk of aggregate wealth and because their portfolio behavior is much more complex than that of the typical household.

Though the foregoing arguments may seem to provide a compelling rationale for studying the portfolios of the rich, there has been little recent academic work in this area. The goal of this paper is to provide a summary of the basic facts about portfolios of wealthy households in the U.S. (and how the facts have changed over time) in a form which allows comparison of their behavior both with the rest of the population in the U.S. and with portfolio behavior among other groups and other countriessurveyed in the recent volume Household Portfolios edited by Guiso, Haliassos, and Jappelli (2001 (Projected)), and to make a preliminary attempt to understand the characteristics that will be required of any model which hopes to be consistent with the observed behavior.

The principal conclusion will be that the most important way in which the portfolios of the rich differ from those of the rest is that the rich hold a much higher proportion of their portfolios in risky investments, with a particularly large concentration of net worth in their own entrepreneurial ventures.

After the empirical conclusions are presented, the paper informally considers how these results relate to theoretical models of portfolio behavior. The starting point will be a standard stochastic version of the Life Cycle/Permanent Income Hypothesis model. That model will prove inadequate, however, because it implies that the rich should look like scaled-up versions of everybody else. They should have neither the extreme wealth-to-income ratios observed in the data, nor the unusual portfolio structures. The goal of the theoretical discussion will be to consider whether any of three potential modifications to the standard model might explain the observed combination of facts.

The first idea is that perhaps there is exogenous, immutable ex ante variation in risk aversion across households. ${ }^{2}$ In that case more risk-tolerant households would

\footnotetext{${ }^{1}$ Bertaut and Starr-McCluer (2001) find that the only kind of financial asset held by more than half of US households is a checking/transactions account.\\
${ }^{2}$ By ex ante we mean a preference difference which exists prior to any saving or portfolio choice decision the household makes, and which is unaffected by the outcomes of such choices.
}
take greater risks and on average would earn higher returns. If owning a private business is the form of economic activity that offers the highest risk and highest return, one might expect that the most risk tolerant households would gravitate toward entrepreneurship, and on average would end up richer (though the failures might end up poorer). ${ }^{3}$ The paper will argue that this story has several defects, ranging from the fact that the empirical evidence fails to find a correlation between wealth growth and initial (expressed) risk aversion to the fact that, taken alone, the story provides neither an explanation for the lack of diversification of entrepreneurial investments nor for the tendency of wealthy households to hold much of their net worth in their own entrepreneurial ventures.

These points lead to the second possibility: that the observed patterns are entirely a consequence of capital market imperfections, as suggested recently by Gentry and Hubbard (1998) and Quadrini (1999). Those authors argue that adverse selection and moral hazard problems require entrepreneurial enterprises to be largely self-financed. They further assume that there is a minimum efficient scale for private enterprises and that this minimum scale is large relative to the wealth of the typical household. The combination of these two assumptions can explain why households with low or moderate wealth or income are less likely to become entrepreneurs. Furthermore, this story requires no differences in tastes among members of the population, and in principle can explain both the high saving rates of the rich and the high portfolio shares in their own entrepreneurial ventures. However, this story too has problems. The first is that, in the absence of differences in preferences between the rich and the rest, the standard model implies that those households who have invested heavily in their own entrepreneurial ventures should try to balance the riskiness of these investments by holding all other assets in very safe forms. Instead, the non-entrepreneurial investments of rich entrepreneurs are much riskier than the portfolios of nonrich nonentrepreneurs. A second problem with this story is that even the model with imperfect capital markets implies that as the rich get old, they eventually begin running down their wealth. In contrast, empirical data reveal no evidence that wealthy elderly households ever begin to run down their wealth.

The final possibility is that the model's assumption about the household utility function needs to be changed in a manner similar to that proposed by Carroll (2000), who simply assumes that wealth enters the utility function as a luxury good in a modified Stone-Geary form. Because Max Weber (1958) argued that a love of wealth for its own sake is the spirit of capitalism, Bakshi and Chen (1996) and Zou (1994) have

\footnotetext{${ }^{3}$ Surprisingly, it is not clear that classical theory supports the proposition that less risk averse individuals will invest a higher proportion of their risky investments in the most-risky activities. See Gollier (2001) for a discussion of the 'mutual fund separation theorem' which implies that the composition of risky assets should be similar whatever the level of risky asset holdings. We will assume that this reflects a limitation of classical theory, rather than a plausible description of behavior.
}
dubbed such models 'capitalist spirit' models. Carroll (2000) proposed this modification to the standard model as a way to explain the high lifetime saving rates of the rich, and argued that many different kinds of behavior, ranging from philanthropic bequest motives to pure greed, would result in a formulation of saving behavior that would be well captured by the modified model. An unanticipated consequence of the model is that it implies that rich households have lower relative risk aversion than the nonrich, which in turn could explain why the rich hold riskier portfolios than the rest, and why high-wealth or high-income young households are more likely to begin entrepreneurial ventures.

The one feature of the data that the 'capitalist spirit' model taken alone cannot explain is the tendency of entrepreneurs to invest largely in their own entrepeneurial ventures, which appears to require some form of capital market imperfection. The paper thus concludes that the main features of the data can probably be explained in a model which combines capital market imperfections of the kind emphasized by Gentry and Hubbard (1998) and Quadrini (1999) with a utility function like that postulated in Carroll (2000).

\section*{2 The Data}
\subsection*{2.1 Portfolios of the Rich}
U.S. survey data on the portfolios of the rich are the best in the world. The 196263 Survey of Financial Characteristics of Consumers (henceforth SFCC) was the first wealth survey to heavily oversample the richest households. The next comprehensive wealth survey was the 1983 Survey of Consumer Finances, which was followed by a 1989 SCF which consisted of a subsample of reinterviewed households from the 1983 survey along with a fresh batch of new households. Since 1989 the SCF has been performed triennially (though with no further panel elements), with the latest survey having been completed in 1998.

The availability of data spanning such a long time period opens up the possibility of studying how portfolios change in response to changes in the economic enviornment. Before examining the data on portfolio structure, therefore, we first present a summary of the taxation and legal changes that we might expect to have had a substantial impact on portfolio structure of wealthy households.

\subsection*{2.1.1 The Tax Environment}
Table 1 summarizes the changes over time in the three aspects of US taxes that are particularly important for the rich. (For information on broader changes in the US tax code see the paper by Poterba (2001)). The first two columns show the statutory top\\
marginal federal tax rate, which declined from 91 percent in 1963 to 39.6 percent in 1993 and thereafter. The second column shows the actual taxes paid as a proportion of their incomes by the richest one percent of households. In spite of the dramatic decline in top marginal rates, the proportion of income paid in taxes has been fairly steady, varying between around 20 and 25 percent over the entire period. This reflects the fact that during the era of high top marginal rates, the tax code was riddled with tax shelters and loopholes that made it possible for almost all rich people to avoid paying the confiscatory top marginal rates on the statute books.

The estate tax is also highly relevant for the rich. The structure of the estate tax is rather complex, but that structure remained largely the same over the period in question. The first $\$ \mathrm{x}$ of an estate is free from estate taxation altogether, where $\$ \mathrm{x}$ is indicated by the column of the table labelled 'exemption.' Above $\$ \mathrm{x}$, taxes begin at a marginal rate of $y$ percent and peak at a top marginal rate of $z$ percent, where $y$ and $z$ are the first and second numbers in the column labelled 'tax range.'

The exclusion for closely held businesses is a mechanism that reduces the reported amount of the value of a closely-held business that is taxable, under the condition that the heir plans to 'actively manage' the business rather than sell it. The marital deduction indicates how much of the estate is taxed when one spouse dies and the estate falls into the hands of the widow or widower. The 100 percent deduction since 1985 means that estates are taxed only when both members of a married couple have died.

The final kind of tax that is relevant to the rich is the gift tax exclusion amount $\$ \mathrm{~g}$, whose value is reported in the last column of the table. This is the amount that each member of the household (husband and wife) can give to any individual (son, daughter, son-in-law, daughter-in-law, grandchildren, etc.) annually without incurring any additional taxes for the recipient or donor.

The table shows that there have been two big changes in the taxes specifically relevant for the rich over the period in question: the large increase in exemption levels for the estate tax in the early 1980s, and the more gradual, but cumulatively very large, decline in top marginal rates. The most important change not captured in the table is probably the abrupt termination of a variety of tax shelters in the 1986 tax reform.

A final feature of the tax code that is relevant for the rich is the 'step-up in basis at death.' The capital gains tax 'basis' for an asset is normally defined as the nominal price at which the asset was bought. However, if the asset has been inherited, then the basis is the nominal valuation of the asset at the time it was inherited. The step up in basis at death provides an incentive for individuals who anticipate leaving a bequest whose value is less than the exemption amount to hold their assets in forms which yield returns disproportionately in the form of capital gains, since capital gains that happen before death are untaxed. (Incentives for the very rich to hold their assets in forms which yield mainly capital gains are smaller because the capital gains do contribute\\
to the valuation of the estate for tax purposes and thus are marginally taxed at the marginal estate tax rate for those who will leave bequests in excess of the exemption amount).

Implications of the tax system for the portfolio structure of the rich are not always easy to determine by examining statutory provisions. For example, the incentive provided by the 'step up in basis at death' to hold assets in forms that yield capital gains depends importantly on the effective marginal rate of taxation on other forms of capital income, which (as discussed above) is not very well proxied by the statutory top marginal rate. The exclusion for closely held businesses does provide an incentive to hold at least a limited absolute amount of the portfolio in the form of closely-held businesses if the individual expects his or her heirs to continue to run the business. However, no marginal incentive to further business ownership is provided once the total amount of wealth held in this form exceeds the exclusion amount. For a more detailed historical analysis of tax policies relevant for the rich in the postwar period, see Brownlee (2000).

\subsection*{2.1.2 Detailed Portfolio Structure}
Our statistical summary of the portfolio structure of the rich begins with Table 2, which provides data on the proportion of the rich (defined here and henceforth as the top one percent of households by net worth) who own any amount of various kinds of assets.

Perhaps the most dramatic change over time in the table is the sharp increase in the proportion of households with defined contribution pension plans. In the 1962-63 SFCC, only 10.1 percent of the rich had any such account, but by 1983 the fraction had already jumped to 65.6, while by 1995 the fraction had reached 78.6 percent. The low percentage in 1962-63 reflects the fact that there was little tax advantage to such plans until the early 1980s, when Individual Retirement Accounts (IRAs) suddenly became available in principle to the whole population, and eligibility for company-based $401(\mathrm{k})$ pension plans was greatly expanded. What is interesting is the speed with which rich households availed themselves of these new options. In contrast, Bertaut and StarrMcCluer (2001) show (Table 3) that only 31 percent of all households had acquired such accounts by the time of the 1983 survey.

Another notable change is that the proportion holding individual stock shares directly has fallen from 84.0 percent in 1962 to 65.0 percent in 1995, while the proportion holding mutual funds has risen from about 24 percent to about 45 percent. This reflects a broad pattern in which households have increasingly decided to hold shares in the form of mutual funds rather than individual stocks. This pattern has not been much studied by economists, although it is interesting because it reflects a convergence of actual behavior toward portfolio theory's recommendation for diversification.

Among the other categories of assets, the largest changes are seen in the holdings of 'other bonds' (primarily corporate bonds), which declined very sharply between 1962 and 1983 and fluctuated substantially between 1983 and 1995. Because nominal interest income is taxable annually while capital gains are taxable only upon realization, the sharp increase in nominal interest rates caused by the acceleration of inflation in the 1960s and 1970s could explain a shift out of interest-bearing assets between the 1963 and 1983. However, there is no obvious tax reason for the fluctuations between 1983 and 1995.

The proportion of the richest households who have equity in a privately held business has fluctuated substantially over the years, from a low of 69.0 percent in 1962-63 to a high of 88.0 percent in 1983. To some extent, fluctuations in this variable may reflect stock market valuations, because after a large increase in stock prices a higher proportion of the wealthy will be rich because of their stock holdings compared with the proportion who are rich because of their holdings of other kinds of assets. (The 1983 SCF was conducted before the bull markets of the 80s and 90s had boosted stock valuations.)

With respect to debt holdings, the proportion of rich households with any debt jumped sharply between the 1962-63 SFCC, when it was 50.2 percent, and the 1983 SCF, when it was 77.9 percent, but exhibited no clear trend thereafter. Among debt categories, the most striking change is the increase in the proportion of households with mortgage debt, from 30.7 percent in 1962 to 52.5 percent in 1995. This likely reflects the fact that mortgage interest remained tax deductible after the 1986 tax reform while other forms of debt lost their deductible status.

On the whole, the striking feature of this table is that the proportion of rich households owning various categories of assets has not changed greatly for most categories of assets - particularly considering that small sample sizes mean that there is inevitably some measurement error in the statistics for any particular year. ${ }^{4}$

Another useful comparison is of the rich to the rest of the population. Average values of ownership shares for the nonrich over the five survey years are presented in the last column of the table. The broadest observation to make here is that rich households are more likely to own virtually every kind of asset. Particularly striking is the discrepancy in the proportion owning equity in a privately held business, which averages about 75 percent for the rich but only 13 percent for the rest of the population. The contrast in ownership of shares in publicly traded companies is only slightly less dramatic: 74 percent versus 16 percent.

Table 3 examines the relative weight of various kinds of assets in the net worth of

\footnotetext{${ }^{4}$ One exception is 'other financial assets,' which had an 89.3 percent owernship rate in the 1962-63 SFCC but much lower rates in the later surveys. This is almost certainly because holdings of cash were included in this grab-bag category in the SFCC but not in the SCF's. In any case, the next table shows that 'other finanical assets' constitute a trivial proportion of net worth in all surveys.
}
the richest households. The table shows that the shift in value from stocks to mutual funds was substantial, but even at the end of the sample in 1995, total net worth in individual shares still remained substantially greater than that in mutual funds. One of the largest shifts over time is in the role of investment real estate, which jumps from 7.4 percent of net worth in 1962-63 to over 20 percent in 1983. Investment real estate continues to constitute more than 20 percent of the portfolio until 1995, when its share drops to 13.1 percent. The jump in investment real estate between the early 1960s and the early 1980s may reflect the prominent role of real estate in tax shelters until the tax reform act of 1986. One would have expected a decline in the value of investment real estate following the repeal of many of these tax shelters in the 1986 tax act, so it is surprising that no decline is manifest until 1995.

Another interesting observation from the table is the small amount of mortgage debt (only 1.1 percent of net worth on average) despite the fact that more than half of the rich have positive amounts of such debt.

Comparing the rich to the rest of the population, again perhaps the most important difference is the importance of business equity for the rich. Such wealth accounts for about 40 percent of total net worth of the rich in 1983 and thereafter, vastly more than its share in the net worth of the typical household. Other differences include the lower total indebtedness of the rich and the much smaller proportion of total wealth tied up in home equity.

\subsection*{2.1.3 Portfolio Structure And Portfolio Theory}
The usual theoretical analysis of portfolio allocation considers the optimal proportion of net worth to invest in 'risky' versus 'safe' assets. This stylized theoretical treatment is conceptually useful but difficult to bring to data, because it is hard to allocate every asset to one of these two categories. Table 4 reflects an effort to find a compromise between the complexity of actual portfolios and the simplicity of theory.

Among financial assets, there are some that are clearly safe (like checking, saving, and money-market accounts) and some that are clearly risky (like stock shares). But other assets are harder to allocate, either because the item itself has an ambiguous status (like long-term government bonds, which are subject to inflation risk but not repayment risk (we hope!)) or because the asset is a composite with unknown proportions of risky and safe assets (like mutual funds which hold both stocks and government bonds). We have allocated all financial assets to one of three categories: Clearly safe, fairly safe, and risky, which can of course be further aggregated into broad measures of safe and risky assets. We have divided nonfinancial assets into the primary residence, investment real estate, business equity, vehicles, and 'other.'

With these definitions, we can construct three definitions of risky assets: A 'narrow' definition, which includes only risky financial assets; a 'broad' definition, which includes\\
clearly and fairly risky financial assets, business equity, and investment real estate; and a 'broadest' definition which adds even the 'fairly safe' assets.

It is apparent from the table that the portfolios of the rich are dramatically more risky than those of the rest of the population. ${ }^{5}$ Across the five surveys the proportion of their portfolios that consisted of broadly risky assets was about 80 percent, compared with an average percentage of only 40 percent for the nonrich households. Examining the data in more detail reveals two key differences between the rich and the rest: the rich hold a much smaller proportion of their wealth in home equity ${ }^{6}$ ( 7.4 percent versus 49.6 percent) and a much larger proportion in business equity and investment real estate (the sum of these two categories is 52.1 percent for the rich versus 26.2 percent for the rest).

\subsection*{2.1.4 Portfolio Diversification and Age Structure}
Another perspective on the portfolios of the rich is presented in Table 5, which provides a census of the portfolio structure of the rich along the three dimensions corresponding to ownership or non-ownership of clearly safe, fairly safe, and risky assets, a total of $2^{3}=8$ different possibilities. In all five survey years, a majority or nearly a majority of the rich held some assets in each of these three categories. This is a sharp contrast to the behavior of the rest of the population, which is much more evenly distributed among the 8 categories but is most heavily concentrated in the region with only safe assets. (See Bertaut and Starr-McCluer (2001) for the data on the rest of the population.)

Finally, Table 6 presents data on ownership rates for risky assets by age of the household head for each of the survey years. ${ }^{7}$ Interestingly, the patterns for ownership rates and for portfolio shares are different: The probability of owning at least some amount of risky assets is monotonically increasing in age, but the proportion of the portfolio composed of 'broad risky' assets rises through the first three age categories (up to age 49 ) but exhibits no clear pattern across the older age groups. ${ }^{8}$ Ownership rates of 'risky' assets show a similar monotonic increase (at least until age $70+$ ), while the portfolio share shows some tendency to decline with age. As King and Leape (1984) argue, the monotonic increase in ownership rates may reflect the accumulation of experience with different assets as the household ages. The reduction in the 'risky' share

\footnotetext{${ }^{5}$ One might wonder whether the differences in risky shares partly reflect age differences between the rich and the rest. However, when the age range for the rich and the rest is restricted to households aged 35-54, the divergence between the rich and the rest is, if anything, even greater. For example, the portfolio share of private business for the age $35-54$ rich is 47.6 versus 17.9 for the age $35-54$ nonrich - a greater discrepancy than the 37.7 versus 14.8 figures in Table 4.\\
${ }^{6}$ Home equity is calculated as the value of primary residence minus mortgage debt.\\
${ }^{7}$ Portfolio shares are for the whole population of the rich, not just for those who own risky assets, i.e. the numbers are not conditional on participation.\\
${ }^{8}$ It is important to recall that these figures may reflect the effects of both cohort and time effects as well as age effects, so the true age effects may differ from the reported numbers.
}
of the portfolio for the 50+ age groups is interesting because it corresponds roughly to the common financial advice to shift assets away from risky forms as retirement approaches (though admittedly no such pattern is evident for the 'broad risky' portfolio share). Note, however, that there is some debate about whether this advice is theoretically sound; furthermore, as shown by the comparative analysis of age profiles of risky investment in several countries in Guiso, Haliassos, and Jappelli (2001 (Projected)), there does not seem to be a consistent pattern to age profiles of the risky portfolio share across countries.

\subsection*{2.1.5 International Evidence on Portfolios of the Rich}
Evidence about portfolios of the rich in other countries is presented in Table 7.The data in this table were provided by the respective country experts who contributed country chapters to the Household Portfolios conference volume referenced in the bibliography. Before describing the results, it is important to emphasize the problems associated with such international comparisons. Probably the greatest problem is that surveys in other countries generally have not made such an intense effort as the SCF does to get a large and representative sample of the very richest households; furthermore, little is known about exactly how participation rates for the wealthy vary across countries. As a result, a table merely presenting data from the top 1 percent of surveyed households across countries might well reflect differences in survey success and methodology more than actual differences in behavior across countries. Our response to this problem is twofold. First, rather than focusing on the top 1 percent, where the variation in participation rates is likely to be very large across countries, we report information about the top 5 percent of households. Second, we strongly discourage direct comparison of portfolio statistics for the 'rich' across countries. Instead, it seems likely to be more reliable simply to examine how the differences between the rich and the rest vary across countries.

Other survey differences also hamper international comparisons. From the standpoint of comparing the results to the predictions of portfolio theory, we would like to be able to divide all assets between safe and risky categories. Unfortunately, the problems in making such allocations are even greater in most other surveys than they are in the SCF. In particular, most surveys collect little or no information about the investment strategies of mutual funds or defined contribution pensions, or about the risk characteristics of other financial assets. Given these problems, we concluded that the most informative feasible exercise was to allow individual country experts to determine, for each asset category, whether there was sufficient information about that category to allocate the asset unambiguously to one of the four levels of riskiness. If not, the analyst was asked to include the asset in the category 'risk characteristics unknown.' An example in the SCF would be a mutual fund which the respondent indicated invested\\
in both stocks and bonds. Because the SCF does not collect any information about the proportion of the fund's value invested in each of these two categories, we included all such mutual fund assets in the 'risk characteristics unknown' category. ${ }^{9}$ Under this strategy, at least the reader can be confident that the assets included in, say, the 'clearly risky' category are indeed all risky.

A final problem is in normalization. Portfolio theory yields predictions about the proportion of the portfolio that should be held in various kinds of assets. Accordingly, table 7 reports the ratio of various kinds of nonfinancial assets and debts to total net worth. It is very important to remember, however, that all of the measurement problems that affect the components of net worth also affect the total. For example, the net value of private business is not measured in the German survey data, and consequently is not included in net worth. Furthermore, the German survey does not provide separate data for the value of the respondent's home and the value of all other real estate owned by that respondent, so the number reported in the table for 'private residence' actually reflects all real estate. Since private business wealth constitutes at least 30 percent of total net worth of the rich in the three countries for which survey data on these components of wealth do exist, and investment real estate is around another 15 percent of net worth, the apparently surprising finding that the gross value of 'private residence' constitutes 88 percent of net worth for the 'rich' German households should not be taken at face value.

Keeping all of these problems in mind, a few conclusions still seem warranted.\\
The most important is probably that in every country the top 5 percent hold a substantially larger proportion of their financial assets in risky forms than do the rest. The difference is smallest in the UK, which may reflect the residual effects of the largescale privatization of the Thatcher years and more recently the demutualization of many formerly cooperative financial enterprises. ${ }^{10}$

Another result common to all countries is that the ratio of debt to net worth is substantially smaller for the rich than for the rest, although the disparity is enormous in some countries (the US) and rather small in others (Italy).

A striking difference across countries is in the breakdown of wealth between financial and nonfinancial forms. The two extremes are the US and Italy. The ratio of nonfinancial to financial wealth for the top 5 percent in the US is about 1.5, while that ratio in Italy is approximately 7. Similar, though less extreme, results hold for the bottom 95 percent of households (where measurement problems are probably somewhat smaller). The Italian country authors indicate that part of the discrepancy probably reflects

\footnotetext{${ }^{9}$ This contrasts with our strategy in Table 4, where we divided such investments $50-50$ between the 'fairly safe' and 'fairly risky' categories.\\
${ }^{10}$ Shares were distributed to depositors, and thus many lower-income households who otherwise owned no shares became shareowners. Research has shown that many lower-wealth households have simply held onto the shares they obtained through demutualizations.
}
systematic severe underestimation of financial assets in Italy. Nonetheless, while the maginitude of the difference may be mismeasured, qualitatively the observation that nonfinancial assets are much more important in Italy than the US is probably true.

A final observation is that there are large differences in the levels of debt held by the bottom 95 percent across countries, ranging from a high of $\$ 36,000$ in the US to a low of only 4290 euro in Italy. This observation reinforces existing research which has found that more highly developed financial markets in the US have allowed much higher levels of borrowing. ${ }^{11}$

\section*{3 Analysis}
It is now time to begin trying to understand the underlying behavioral patterns which give rise to the data reported above. We start by presenting a baseline formal model of saving over the life cycle, to which we will add a portfolio choice decision.

\subsection*{3.1 The Basic Stochastic Life Cycle Model}
The following model is what I will henceforth characterize as the basic stochastic life cycle model. The consumer's goal is to


\begin{equation*}
\max \quad \sum_{s=t}^{T} \beta^{s-t} \mathcal{D}_{t, s} u\left(C_{t}\right) \tag{1}
\end{equation*}


where $u(C)$ is a constant relative risk aversion utility function $u(C)=c^{1-\rho} /(1-\rho), \beta$ is the (constant) geometric discount factor, and $\mathcal{D}_{t, s}=\prod_{h=t}^{s-1}\left(1-d_{h}\right)$ is the probability that the consumer will not die between periods $t$ and $s$ ( $\mathcal{D}_{t, t}$ is defined to be $1 ; d_{t}$ is the probability of death between period $t$ and $t+1$ ).

The maximization is of course subject to constraints. In particular, if, following Deaton (1991), we define $X_{t}$ as 'cash-on-hand' at time $t$, the sum of wealth and current income, then the consumer faces a budget constraint of the form

$$
X_{t+1}=R_{t+1} S_{t}+Y_{t+1}
$$

\footnotetext{${ }^{11}$ Italy is a particularly interesting case. Until recently, the minimum down payment on a home mortgage in Italy was on the order of 50 percent, while 5 percent down payment mortgages have been common in the U.S. for at least a decade. Furthermore, the legal system in Italy makes reposession of property extremely difficult and time consuming. Thus, many Italians cannot afford to buy a house, and those who do buy end up borrowing much less. The high value of nonfinancial assets (mainly housing wealth) relative to financial is probably largely attributable to these features of the Italian financial system.
}
where $S_{t}=X_{t}-C_{t}$ is the portion of last period's resources the consumer did not spend, $R_{t+1}$ is the gross rate of return earned between $t$ and $t+1$, and $Y_{t+1}$ is the noncapital income the consumer earns in period $t+1$.

Assume that the consumer's noncapital income in each period is given by their permanent income $P_{t}$ multiplied by a mean-one transitory shock, $E_{t}\left[\tilde{\epsilon}_{t+1}\right]=1$, and assume that permanent income grows at rate $G_{t}$ between periods, but is also buffeted by a mean-one shock, $P_{t+1}=G_{t+1} P_{t} \eta_{t+1}$ such that $E_{t}\left[\tilde{\eta}_{t+1}\right]=1$, where our notational convention is that a variable inside an expectations operator whose value is unknown as of the time at which the expectation is taken has a $\sim$ over it.

Given these assumptions, the consumer's choices are influenced by only two state variables at a given point in time: the level of the consumer's assets $X_{t}$ and the level of permanent income, $P_{t}$. As usual, the problem can be rewritten in recursive form with a value function $V_{t}\left(X_{t}, P_{t}\right)$. Written out fully in this form, the consumer's problem is

\[
\begin{array}{rll}
V_{t}\left(X_{t}, P_{t}\right) & = & \max _{\left\{C_{t}\right\}} u\left(C_{t}\right)+\beta \mathcal{D}_{t, t+1} E_{t}\left[V_{t+1}\left(\tilde{X}_{t+1}, \tilde{P}_{t+1}\right)\right] \\
& \text { such that }  \tag{2}\\
S_{t} & = & X_{t}-C_{t} \\
X_{t+1} & = & R_{t+1} S_{t}+Y_{t+1} \\
Y_{t+1} & = & P_{t+1} \epsilon_{t+1} \\
P_{t+1} & = & G_{t} P_{t} \eta_{t+1}
\end{array}
\]

\subsection*{3.2 The Saving Behavior of the Rich}
Within the last decade, advances in computer speed and numerical methods have finally allowed economists to solve life cycle consumption/saving problems like that presented above with serious uncertainty and realistic utility (see, in particular, Hubbard, Skinner, and Zeldes (1994); Huggett (1996); Carroll (1997); and the references therein). I have argued elsewhere (Carroll (1997)) that the implications of these models fit the available evidence on the consumption/saving behavior of the typical household reasonably well, certainly much better than the old Certainty Equivalent (CEQ) models did.

However, another finding from this line of research has been that the model is unable to account for the very high concentrations of wealth at the top of the distribution.

\subsection*{3.2.1 How Rich Are They?}
Figure 1 shows the ratio of wealth to permanent income ${ }^{12}$ by age for the population as a whole and for the households in the richest one percent by age category from the 1992 and 1995 SCFs. Also plotted for comparison is the level of the wealth to income ratio at the top 1 percent implied by a standard life cycle model of saving similar to that in Carroll (1997) or Hubbard, Skinner, and Zeldes (1994). (Specifically, it is the Carroll model with HSZ 'baseline' parameter values). The richest one percent are much richer than implied by the life cycle model. In addition, the figure plots the age profile of the 99th percentile that would be implied by the HSZ model if it were assumed that households do not discount future utility at all. The figure shows that even with such patient households, the model remains far short of predicting the observed wealth to income ratios at the 99th percentile. ${ }^{13}$

This finding is reconfirmed in a recent paper by Engen, Gale, and Uccello (1999), who do a very careful job of modelling pension arrangements, tax issues, and other institutional details neglected in Carroll (2000) and also find that the wealth-to-income ratios at the top part of the income distribution are much greater than predicted by a life cycle dynamic stochastic optimization model, even with a time preference rate of zero.

\subsection*{3.2.2 How Do They Spend It All?}
They don't.\\
In the 1989, 1992, and 1995 SCFs, households were asked whether their spending usually exceeds their income, and whether their spending exceeded their income in the previous year. In order to run down their wealth, households obviously must eventually spend more than their income. Yet only five percent of the rich elderly households in the SCF answered that their spending usually exceeded their income.

More evidence is presented in Figure 2, which shows the levels of wealth by age for the elderly in the 1992 and 1995 SCFs. There is no evidence in this figure that wealth is declining for this population; indeed, if anything it seems to be increasing, ${ }^{14}$ consistent with the answers that the rich elderly give to the questions about whether they are spending more than their incomes. The implication is that most of the wealth which we observe them holding will still be around at death. This is clearly a problem for any model in which the only purpose in saving is to provide for one's own future

\footnotetext{${ }^{12}$ SCF respondents are asked whether their total income this year was above normal, about normal, or below normal. Following Friedman (1957), I define permanent income as the level of income the household would normally receive.\\
${ }^{13}$ This figure is reproduced from Carroll (2000).\\
${ }^{14}$ This is in effect a smoothed profile of wealth by age adjusted for cohort effects; see Carroll (2000) for methodological details.
}
consumption.\\
This crude evidence is backed up by a study by Auten and Joulfaian (1996) which finds that the elasticity of bequests with respect to lifetime resources is well in excess of one (their point estimate is 1.3 ). See Carroll (2000) for a summary of further evidence that, far from spending their wealth down, the rich elderly continue to save.

\subsection*{3.3 Adding Portfolio Choice}
Recently, a wave of papers (Bertaut and Halaissos (1997); Fratantoni (1998);Gakidis (1998); Cocco, Gomes, and Maenhout (1998); and Hochgurtel (1998)) has examined the predictions of stochastic life cycle models of the kind considered above when households facing labor income risk are allowed to choose freely between investing in a low-return safe asset and investing in risky assets parameterized to resemble the returns yielded by equity investments in the past.

The only modification to the formal optimization problem presented above necessary to allow portfolio choice is to designate $R_{t+1}$ as the portfolio-weighted return, which will depend on the proportion of the portfolio that is allocated to the safe and the risky assets, and on the rate of return on the risky asset between $t$ and $t+1$. Call the proportion of the portfolio invested in the risky asset ('stocks') $w_{s, t}$ (where $w$ is mnemonic for the portfolio 'weight'), and ( $1-w_{s, t}$ ) is the portion invested in the safe asset. If the return on stocks between $t$ and $t+1$ is $R_{s, t+1}$, the portfolio-weighted return on the consumer's savings will be $R\left(1-w_{s, t}\right)+R_{s, t} w_{s, t}$.

However, even without solving a model of this type formally, it is clear that such models will not be able to explain the empirical differences between the portfolio behavior of the rich and the behavior of the rest of the population, because when the utility function is in the CRRA class, problems of this type are homothetic. That is, there is no systematic difference in the behavior of households at different levels of lifetime permanent income. Hence, such models provide no means to explain the very large differences between the rich and the rest in saving and portfolio behavior documented above.

\subsection*{3.4 Three Possible Modifications}
There are at least three ways one might consider modifying the model in hopes of explaining the apparent nonhomotheticity of saving and portfolio behavior.

\subsection*{3.4.1 Heterogeneity in Risk Tolerance}
The first is simply to allow for exogenous, immutable ex ante heterogeneity in risk tolerance across members of the population. Formally, rather than assuming that\\
all households have the same value of $\rho$, we can assume that each household has an idiosyncratic, specific $\rho_{i}$.

The effect of this would be to allow households with low values of $\rho$ (high risk tolerance) to choose highly risky but high-expected-return portfolios. On average, the risk-tolerant households would be rewarded with higher returns and would therefore end up richer than the rest of the population. Thus, the rich would be disproportionately risk-lovers, and would therefore have riskier portfolios than the rest. As shorthand, I will call this the 'preference heterogeneity' story henceforth.

\subsection*{3.4.2 Capital Market Imperfections}
A second possibility is to follow Gentry and Hubbard (1998) and Quadrini (1999) in assuming that there are important imperfections in capital markets which 1) require entrepreneurial investment to be largely self-financed; 2) imply that entrepreneurial investment has a higher return than investments made on open capital markets; and 3) require a large minimum scale of investment. As those authors show, the combination of these three assumptions can yield an implication that portfolios of higher wealth or higher income households will be much more heavily weighted toward entrepreneurial investments, and that rich households with business equity have higher than average saving rates (under the further assumption that the intertemporal elasticity of substitution is high which means that they take advantage of the high returns that are available to them by saving more). I will refer to this theory as the 'capital market imperfections' story.

\subsection*{3.4.3 Bequests as a Luxury Good}
A final possibility is to change the assumption about the lifetime utility function. Carroll (2000) proposes adding a 'joy of giving' bequest motive of the form $B(S)$ in a modified Stone-Geary form, ${ }^{15}$

$$
B(S)=\frac{(S+\gamma)^{1-\alpha}}{1-\alpha}
$$

Carroll (2000) shows that if one assumes that $\alpha<\rho$ then wealth will be a 'luxury good' in the sense that as lifetime resources rise, a larger proportion of those resources is devoted to $S_{T}$. In the limit as lifetime resources approach infinity, the proportion of

\footnotetext{${ }^{15}$ It might seem that a 'joy of giving' bequest motive and a 'dynastic bequest motive' of the type considered by Barro (1974) would be virtually indistinguishable, but it turns out that there are several important differences. For example, the dynastic bequest model collapses to a standard life cycle model for households with no offspring, yet empirical evidence suggests that the rich childless elderly continue to save. See Carroll (2000) for more arguments that the 'joy of giving' bequest motive fits the data better.
}
resources devoted to the bequest approaches 1 . The other salient feature of the model is that if $\gamma>0$ there will be a 'cutoff' level of lifetime resources such that households poorer than the cutoff will leave no bequest at all. Thus the model is capable of matching the crude stylized fact that low-income people tend to leave no bequests, and also captures the fact (from Auten and Joulfaian (1996)) that among those who leave bequests, the elasticity of lifetime bequests with respect to lifetime income is greater than one.

In this paper the assumption is that one receives utility from the contemplation of the potential bequest in proportion to the probability that death (and the bequest) will occur. Thus Bellman's equation is modified to

$$
V_{t}\left(X_{t}, P_{t}\right)=\max _{\left\{C_{t}, w_{s, t}\right\}} u\left(C_{t}\right)+\beta\left(1-d_{t}\right) E_{t}\left[V_{t+1}\left(\tilde{X}_{t+1}, \tilde{P}_{t+1}\right)\right]+d_{t} B\left(S_{t}\right)
$$

and the transition equations for the state variables are unchanged.\\
While it is obvious how this model might help to explain the high saving rates of the rich, it is not so obvious why it might help explain the high degree of riskiniess of their portfolios. It turns out, however, that precisely the same assumption which implies that bequests are a luxury good also implies that households are less risk-averse with respect to gambles over bequests than with respect to gambles over consumption. ${ }^{16}$ That assumption is that the exponent on the utility-from-bequests function $\alpha$ must be less than the exponent on the utility from consumption $\rho$. This implies that the marginal utility from bequests declines more slowly than the marginal utility from consumption and thus as wealth rises more and more of it is devoted to bequests rather than consumption. However, the traditional interpretation of exponents like $\rho$ and $\alpha$ in utility functions of this class is as coefficients of relative risk aversion, so the assumption that bequests are a luxury good has the immediate implication of less risk aversion with respect to bequest gambles than consumption gambles!

Following Max Weber as recently interpreted by Zou (1994) and Bakshi and Chen (1996), I will henceforth call this the "Capitalist Spirit" model.

\subsection*{3.5 Distinguishing the Three Models}
All of these theories can in principle explain the basic facts that the portfolios of the rich are disproportionately risky and that investments in closely-held businesses are a disproportionate share of the portfolios of the rich. This section attempts to distinguish between the three theories on the basis of other kinds of evidence.

We begin with some direct evidence that there are substantial differences in the risk preferences of the rich compared with the rest of the population. Table 8 reports

\footnotetext{${ }^{16}$ The presence of the $\gamma$ term in $B(S)$ implies increasing relative risk aversion as bequest gambles get larger. However, this does not alter the fact that risk aversion with respect to gambles over bequests is always less than risk aversion with respect to gambles over consumption.
}
the results of a direct question SCF respondents are asked about their risk tolerance. Specifically, the respondents are asked

Which of the statements on this page comes closest to the amount of financial risk that you (and your [husband/wife/partner]) are willing to take when you save or make investments?

\begin{enumerate}
  \item TAKE SUBSTANTIAL FINANCIAL RISKS EXPECTING TO EARN SUBSTANTIAL RETURNS.
  \item TAKE ABOVE AVERAGE FINANCIAL RISKS EXPECTING TO EARN ABOVE AVERAGE RETURNS.
  \item TAKE AVERAGE FINANCIAL RISKS EXPECTING TO EARN AVERAGE RETURNS.
  \item NOT WILLING TO TAKE ANY FINANCIAL RISKS.
\end{enumerate}

For 1992 and 1995, the table reports the mean values of the response and the percent of households reporting that they are not willing to take any financial risks, by permanent income and net worth percentile. ${ }^{17}$ The table shows that occupants of the highest permanent income and net worth brackets are notably more likely to express a willingness to accept above-average risk in exchange for above-average returns. Even more dramatic is the difference between the proportion of the rich and of the rest who express themselves as 'not willing to take any financial risks.' Among the richest 1 percent by wealth, less than ten percent express such extreme risk aversion; among the bottom 80 percent, nearly half express this sentiment.

Although economists have traditionally dismissed answers to survey questions of this type as meaningless, a recent literature (with contributions by Kahneman, Wakker and Sarin (1997), Oswald (1997), Barsky, Juster, Kimball, and Shapiro (1997), and Ng (1997)) has argued forcefully that answers to questions about preferences can provide reliable and useful information. Thus, these answers should be taken as serious evidence that the rich are more risk tolerant than the rest.

However, the table does not answer the question of the direction of causality between risk preference and wealth. It is possible, as the preference heterogeneity story would have it, that high risk tolerance leads to wealth, but it is equally possible that there is causality from wealth to risk tolerance.

One piece of existing evidence that is suggestive of causality from wealth to risk tolerance is the finding by Holtz-Eakin, Rosen, and Joulfaian (1994) that the receipt of an inheritance substantially increases the probability that the recipient will start an

\footnotetext{${ }^{17}$ Our method of identifying permanent income is simple: we restrict the sample to households who reported that their income in the survey year was 'about normal.' Thus we are employing Friedman's original definition of permanent income, rather than modern definitions as the annuity value of human and nonhuman wealth.
}
entrepreneurial venture. Their interpretation is that because the inheritors presumably knew that they would eventually inherit, their failure to start the entrepreneurial venture in advance of the receipt of the inheritance demonstrates the presence of liquidity constraints. An alternative interpretation is that the increase in disposable wealth increases the household's risk tolerance enough for them to become willing to take the risk of starting an entrepreneurial venture. ${ }^{18}$

The ideal experiment to answer the causality question would be to exogenously dump a large amount of wealth on a random sample of households and examine the effect both on their expressed risk preferences and on their risk-taking behavior. The closest approximation to this ideal experiment in an available dataset is the receipt of unexpected inheritances between the 1983 and 1989 panels of the SCF.

Table 9 presents the results of a simple regression analysis of the change in risk aversion between 1983 and 1989 on the size of inheritances received between the two surveys, using the numerical answer to the survey question about risk attitudes as the measure of risk aversion. That is, defining RISKAV83 as the 1983 answer to the risk aversion question and RISKAV89 as the 1989 answer, we define DRISKAV $=$ RISKAV89-RISKAV83 and regress DRISKAV on a measure of the size of inheritances received and a set of control variables. ${ }^{19}$ Specifically, LINH is the $\log$ of the value of inheritances, and the control variables in the weighted regression are the same as the variables used by Gentry and Hubbard (1998) in their extensive investigation of entrepreneurship using these data.

The coefficient on LINH is overwhelmingly statistically signficant and negative, indicating that larger inheritances produce a greater decline in risk aversion. Recall that the simple preference heterogeneity story was one in which individuals enter the workforce with a built-in level of risk aversion which was unchanging through the lifetime. If we interpret the RISKAV83 and RISKAV89 variables as measures of this risk aversion, the results in Table 9 constitute a direct rejection of this story. Indeed, almost half of households whose composition is unchanged report a different value of RISKAV89 than RISKAV83, and given that a large proportion of the change can be explained ex post via regressions like that reported in Table 9, it is clear that these changes do not merely reflect measurement error.

One potential problem with this experiment is that inheritances may be anticipated. If so, the recipient might take the prospective inheritance into account in formulating risk attitudes even before actual receipt. However, if this were true, a regression of the

\footnotetext{${ }^{18}$ Note that capital market imperfections or uncertainty about the timing and/or size of the bequest are still required; without either uncertainty or imperfections the household's effective wealth would not change when a perfectly anticipated bequest was received.\\
${ }^{19}$ The sample is restricted to households whose composition did not change between the two survey years, in order to ensure that changes in risk aversion really reflect changes in the attitudes of the same individual(s).
}
change in risk preferences on the size of inheritances received would find a coefficient of zero, and so the fact that we found a highly significantly negative coefficient despite this bias only strengthens the case that changes in wealth affect risk aversion. Indeed, when we restrict the sample (in column 2) to those households who said in 1983 that they did not expect ever to receive a substantial inheritance (and who presumably were surprised when they did), the coefficient estimate is a bit larger (though the difference is not statistically significant).

Unfortunately, there is a more serious problem with the experiment: The survey question cannot necessarily be interpreted as revealing the respondent's underlying coefficient of relative risk aversion. Instead, the question is about the respondent's willingness to bear financial risk, and economic theory informs us that willingness to bear financial risk should depend upon a great many factors in addition to an agent's raw coefficient of relative risk aversion. In particular, what should matter is the expected coefficient of relative risk aversion for the future period's value function, which may depend, for example, on whether the consumer anticipates possibly being liquidity constrained in that future period. However, one conclusion from the recent work on portfolio theory cited above is that the proportion of the portfolio invested in the risky asset should decline in the level of current-period cash-on-hand. The reason for this counterintuitive result is that when there is little financial wealth, virtually all of future consumption will be financed by labor income, and so adding a small financial risk has very little effect on overall consumption risk, so the agent is willing to invest a large proportion of her modest portfolio in the risky financial asset (this argument relies on an implicit assumption that the correlation between financial risk and labor income risk is low, as Cocco et. al. (1998) show it is). As wealth grows large, however, the proportion of future consumption to be financed out of that wealth also grows large, and thus willingness to bear additional financial risk declines (see Cocco, Gomes, and Maenhout (1998) for a fuller discussion of these issues). Thus, appropriately calibrated portfolio theory implies that we would expect to see a declining willingness to bear financial risk as wealth increased, rather than the reverse as indicated in the table.

Many economists remain uncomfortable with using survey measures like the SCF risk attitudes question. However, even if the results of Table 9 are set aside, there are several other problems with the preference heterogeneity story as a complete explanation for the observed pattern of facts.

In principle, the preference heterogeneity story can indeed explain the large share of business equity in the portfolios of the richest households, under the assumption that private business investments bear the highest risk and the highest return among the categories of assets available. This assumption is plausible and therefore not problematic. However, several other features of the entrepreneurial investments of the rich are problematic for this theory.

First, entrepreneurial investments of the rich are highly undiversified. If the rich\\
are even slightly risk-averse, elementary portfolio theory under perfect capital markets implies that the optimal strategy is to invest a tiny amount in each of a large number of entrepreneurial ventures in order to diversify the idiosyncratic risk. Table 10 shows that instead, among the rich households with any private business equity, over 80 percent of that equity is in a single entreprenurial venture, while the three largest entreprenurial investments account for 94 percent of entrepreneurial wealth (a similar pattern holds for nonrich entrepreneurs). Furthermore, for rich entrepreneurial households, almost half of all income comes directly from business enterprises in which the household has an ownership stake. Failure of the business would wipe out not only the asset value of the business, but also the business-derived income, and thus the total riskiniess of business ownership is even greater than appears from the share of business equity in total net worth. This means the incentive for diversification is even stronger.

The next problem for the preference heterogeneity theory is that it provides no explanation for the fact that the great majority of entrepreneurial wealth is in enterprises in which a member of the household has an active management role. Table 10 shows that 85 percent of all entrepreneurial wealth is held in such 'actively managed' businesses. Again, with perfect capital markets, management should be completely detached from ownership to diversify idiosyncratic risk.

A final problem is that the preference heterogeneity story provides no explanation for the failure of the elderly rich to spend down their assets. Indeed, because risk tolerance is positively correlated with the intertemporal elasticity of substitution in models with time-separable preferences, we should actually expect the rich to be running down their wealth faster than the non-rich if the only difference in preferences between the rich and non-rich is in their degree of risk tolerance.

Given that several of the preceding arguments imply that the preference heterogeneity story also requires some form of capital market imperfections in order to explain the data, it is interesting to examine whether capital market imperfections by themselves might do the trick.

The central requirement of any story based purely on capital market imperfections is that business ownership must yield higher-than-market rates of return. Unfortunately, the economics and business literatures do not appear to contain credible estimates of the average rate of return on closely-held business ventures. Suppose for the moment that we accept on faith the proposition that closely-held business ventures earn a higher rate of return (in exchange for higher risk) than is available on open capital markets, and that such ventures must be substantially self-financed for moral hazard or adverse selection reasons. By themselves, and in the absence of preference heterogeneity, these assumptions cannot explain the positive correlation between the level of initial labor income or initial wealth and the propensity to start businesses documented by Quadrini (1999) and Gentry and Hubbard (1998). The problem is that these arguments should apply with just as much force to very small business ventures (which\\
can be financed without large initial wealth or income) as to larger ones: As anyone who has read the novel $A$ Confederacy of Dunces knows, there are principal/agent and moral hazard problems even for a hot dog vendor.

Gentry and Hubbard (1998) address this problem by simply assuming that there is a minimum efficient scale for business enterprises which is large relative to the resources of the median household, but this approach is insufficient to explain the data because the richest households would have wealth vastly greater than any fixed minimum efficient scale and therefore would have no need to tie up more than a trivial fraction of their total net worth in any single business enterprise. Quadrini (1999) deals with this problem by postulating a 'ladder' of business opportunities at ever-rising minimum efficient scales, so that no matter how rich the household becomes there is always an opportunity to jump up to an even-higher rung on the ladder.

Even if we were to accept the story that there is a complicated ladder of minimum efficient scales of business operation a la Quadrini, the capital market imperfections story still faces three problems. First, it provides no explanation for the failure of the rich elderly eventually to begin running down their wealth. Second, if the risk preferences of the rich were similar to those of the rest, the extra risk associated with their entrepreneurial wealth should induce them to try to minimize the riskiness of the remainder of their portfolio. However, Table 11 shows that the financial asset holdings of rich households who have a substantial fraction of their net worth tied up in business equity are actually considerably riskier than the financial asset holdings of the rest of the population (although less risky than the financial investments of the rich nonentrepreneurs). Finally, the results in Table 8 strongly suggest that the rich, whether entrepreneurs or nonentrepreneurs, are much more risk tolerant than the rest of the population, and capital market imperfections alone can explain neither this nor the finding in Table 9 that increases in wealth produce reductions in reported risk aversion.

It is now time to consider whether the 'capitalist spirit' model can explain the overall pattern of facts. Recall that this model assumed that bequests are a luxury good, with the corollary implication that households are less risk averse with respect to risks to their bequests than with respect to risks to their consumption. Since the luxury good aspect of bequests implies that as a household becomes richer, it plans to devote more and more of its resources to the bequest, the model also implies that risk aversion declines in the level of wealth. Thus the capitalist spirit model is consistent with the results on self-reported risk attitudes, as well as with the risky portfolio structure for the rich, and with the evidence that receipt of inheritances reduces risk aversion (at least if we assume that those inheritances were not perfectly anticipated). It also can explain why higher income or higher net worth households are more likely to invest in risky entrepreneurial ventures. Finally, it can explain the failure of the elderly rich to run down their assets before death (indeed, this is the empirical fact that the model was\\
developed to explain; its ability to explain the other empirical patterns documented here was not anticipated in the original statement of the model).

However, if capital markets were perfect, rich households would still have every incentive to diversify the idiosyncratic component of their entrepreneurial investments by holding small shares of many entrepreneurial ventures. Their failure to do so is presumably explained by capital market imperfections. ${ }^{20}$ Note, however, that one attractive feature of a model which combines capital market imperfections with the 'capitalist spirit' utility function is that it is possible to dispense with the awkward assumption of a 'ladder' of minimum efficient scales which was necessary in the basic model of capital market imperfections in order to explain the data. This makes the analysis of such models considerably more tractable, transparent, and plausible.

\section*{4 Conclusions}
The standard model of household behavior implies that the rich are just like scaled-up versions of everybody else, including in their portfolio allocation patterns. The data summarized in this paper contradict that assertion both for the US since 1963 and for the other countries included in this survey.

The most important differences between the portfolios of the rich and the rest are the much higher proportion of their assets that the rich hold in risky forms, and their much higher propensity to be involved in entrepreneurial activities and to hold much of their net worth in the form of their own entrepreneurial ventures. Several different features of the data point to a conclusion that relative risk aversion is a decreasing function of wealth. Other features, particularly the concentration of the wealth of the rich in their own entrepreneurial ventures, suggest that capital market imperfections are also important. But it appears that most of the features of the data can be explained by assuming both that bequests are a luxury good and that capital market imperfections require entrepreneurial enterprises to be largely self-financed and self-managed.

This is not to say that there could not also be exogenous differences in risk aversion across households; just that the case does not appear to be strong that such differences are necessary to explain the differences between the portfolios of the rich and the rest.

\footnotetext{${ }^{20}$ In discussing this paper, Marco Pagano suggested that entrepreneurs may obtain utility directly from the ownership and consequent control over their entrepreneurial ventures. This would lead to a preferences-based theory of the nondiversification of entrepreneurial investment. However, in order to explain the overall pattern of facts, it would still be necessary to modify the utility function to put wealth in the utility function in some form, since not all of the rich are entrepreneurs. Because there is substantial independent evidence of capital market imperfections, it seems preferable to stick with a story which explains the facts via an assumption of capital market imperfections plus a single change in the utility function rather than with perfect capital markets plus two changes in the utility function.
}\section*{References}
Auten, Gerald, and David Joulfaian (1996): "Charitable Contributions and Intergenerational Transfers," Journal of Public Economics, 59(1), 55-68.

Bakshi, Gurdip, and Zhiwu Chen (1996): "The Spirit of Capitalism and StockMarket Prices," American Economic Review, 86(1), 133-157.

Barro, Robert J. (1974): "Are Government Bonds Net Worth?," Journal of Political Economy, 82, 1095-117.

Barsky, Robert B., F. Thomas Juster, Miles S. Kimball, and Matthew D. Shapiro (1997): "Preference Parameters and Behavioral Heterogeneity: An Experimental Approach in the Health and Retirement Survey," Quarterly Journal of Economics, CXII(2), 537-580.

Bertaut, Carol, and Martha Starr-McCluer (2001): "Household Portfolios in the United States," in Household Portfolios, ed. by Luigi Guiso, Michael Haliassos, and Tullio Jappelli. MIT Press.

Bertaut, Carol C., and Michael Haliassos (1997): "Precautionary portfolio behavior from a life-cycle perspective," Journal of Economic Dynamics And Control, 21(8-9), 1511-1542.

Booth, Mark (1998): "Projecting Federal Tax Revenues and the Effect of Changes in Tax Law," Congressional Budget Office.

Brownlee, W. Elliot (2000): "Historical Perspectives on U.S. Tax Policy Toward the Rich," in Does Atlas Shrug? The Economic Consequences of Taxing the Rich, ed. by Joel B. Slemrod. Forthcoming, Harvard University Press.

Carroll, Christopher D. (1997): "Buffer-Stock Saving and the Life Cycle/Permanent Income Hypothesis," Quarterly Journal of Economics, CXII(1), 1-56, \href{http://www.econ.jhu.edu/people/ccarroll/BSLCPIH.pdf}{http://www.econ.jhu.edu/people/ccarroll/BSLCPIH.pdf} .

\begin{itemize}
  \item (2000): "Why Do the Rich Save So Much?," in Does Atlas Shrug? The Economic Consequences of Taxing the Rich, ed. by Joel B. Slemrod. Forthcoming, Harvard University Press, \href{http://www.econ.jhu.edu/people/ccarroll/Why.pdf}{http://www.econ.jhu.edu/people/ccarroll/Why.pdf} .
\end{itemize}

Cocco, Joao, Francisco J. Gomes, and Pascal J. Maenhout (1998): "Consumption and Portfolio Choice Over the Life Cycle," Manuscript, Harvard University.

Deaton, Angus S. (1991): "Saving and Liquidity Constraints," Econometrica, 59, 1221-1248.

Dynan, Karen E., Jonathan S. Skinner, and Stephen P. Zeldes (1996): "Do the Rich Save More?," Manuscript, Board of Governors of the Federal Reserve System.

Engen, Eric, William Gale, and Cori Uccello (1999): "The Adequacy of Retirement Saving," Brookings Papers on Economic Activity, 1999(2).

Fratantoni, Michael C. (1998): "Income Uncertainty and the Equity Premium Puzzle," Manuscript, Johns Hopkins University.

Friedman, Milton A. (1957): A Theory of the Consumption Function. Princeton University Press.

Gakidis, Haralobos E. (1998): "Stocks for the Old? Earnings Uncertainty and Life-Cycle Portfolio Choice," Manuscript, Massachusetts Institute of Technology.

Gentry, William M., and R. Glenn Hubbard (1998): "Why Do the Wealthy Save So Much? Saving and Investment Decisions of Entrepreneurs," Manuscript, Columbia University.

Gollier, Christian (2001): "What Does the Classical Theory Have to Say About Household Portfolios?," in Household Portfolios, ed. by Luigi Guiso, Michael Haliassos, and Tullio Jappelli. MIT Press.

Guiso, Luigi, Michael Haliassos, and Tullio Jappelli (eds.) (2001 (Projected)): Household Portfolios. MIT Press, Cambridge, MA.

Hochgurtel, Stefan (1998): "A Buffer Stock Model with Portfolio Choice: Implications of Income Risk and Liquidity Constraints," Manuscript, Uppsala University.

Holtz-Eakin, Douglas, Harvey S. Rosen, and David Joulfaian (1994): "Entrepreneurial Decisions and Liquidity Constraints," RAND Journal of Economics, pp. 334-347.

Hubbard, R. Glenn, Jonathan S. Skinner, and Stephen P. Zeldes (1994): "The Importance of Precautionary Motives for Explaining Individual and Aggregate Saving," in The Carnegie-Rochester Conference Series on Public Policy, ed. by Allan H. Meltzer, and Charles I. Plosser, vol. 40, pp. 59-126.

Huggett, Mark (1996): "Wealth Distribution in Life Cycle Economies," Journal of Monetary Economics, 38(3), 469-494.

Johnson, Barry, and Marthat Britton Eller (1998): "Federal Taxation of Inheritance and Wealth Transfers," in Turing Administrative Systems into Information Systems 1996-1997. Internal Revenue Service, Washington, DC.

Joulfaian, David (1998): "The Federal Estate and Gift Tax: Description, Profile of Taxpayers, and Economic Consequences," Department of the Treasury, Office of Tax Analysis, OTA Paper 80.

Kahneman, Danny, Peter P. Wakker, and Rakesh Sarin (1997): "Back to Bentham? Explorations of Experienced Utility," Quarterly Journal of Economics, 112, 375-406.

Kasten, Richard, Frank Sammartino, and David Weiner (1998): "Estimates of Federal Tax Liabilities for Individuals and Families By Income Category and Family Type for 1995 and 1999," Congressional Budget Office, Report to the House and Senate Budget Committees and the House Ways and Means Committee, Avilable at \href{http://www.cbo.gov/showdoc.cfm?index=527&sequence=0&from=1}{http://www.cbo.gov/showdoc.cfm?index=527\&sequence=0\&from=1} as of May 23, 2000.

King, Mervyn, and Jonathan Leape (1984): "Asset Accumulation, Information, and the Life Cycle," NBER Working Paper No. 2392.

Ng, Y. K. (1997): "A Case for Happiness, Cardinalism, and Interpersonal Comparability," Economic Journal, 107, 1848-1858.

Oswald, Andrew J. (1997): "Happiness and Economic Performance," Economic Journal, 107, 1815-1831.

Poterba, James M. (2001): "Taxation and Portfolio Structure: Issues and Implications," in Household Portfolios, ed. by Luigi Guiso, Michael Haliassos, and Tullio Jappelli. MIT Press.

Quadrini, Vincenzo (1999): "Entrepreneurship, Saving, and Social Mobility," Journal of Economic Dynamics, 3.

Slemrod, Joel (1994): Tax Progressivity and Income Inequality. Cambridge University Press, New York.

Weber, Max M. (1958): The Protestant Ethic and the Spirit of Capitalism. Charles Scribner and Sons, New York.

Zou, Heng-Fu (1994): "The 'Spirit of Capitalism' and Long-Run Growth," European Journal of Political Economy, 10(2), 279-93.

\begin{table}[h]
\begin{center}
\captionsetup{labelformat=empty}
\caption{Table 1: Major Features of the Tax Code Relevant for the Rich}
\begin{tabular}{|l|l|l|l|l|l|l|l|}
\hline
\multirow[t]{2}{*}{Year} & \multicolumn{2}{|c|}{Top 1\% by income} & \multicolumn{4}{|c|}{Estate tax} & Gift tax \\
\hline
 & Marginal rate & Effective rate ${ }^{1}$ & Tax range & Exemption & Exclusion for closely held business & Marital deduction & Annual exclusion ${ }^{\mathbf{2}}$ \\
\hline
1963 & 91\% & 24.6\% & 3-77\% & \$268,581 & NA & 50\% & $\$ 13,429^{3}$ \\
\hline
1977 & 70\% & 27.8\% & 18-70\% & \$295,971 & \$1,226,400 & 50\% or \$613,200 & \$7,358 \\
\hline
1980 & 70\% & 23.9\% & 18-70\% & \$312,071 & \$965,790 & 50\% or \$482,895 & \$5,794 \\
\hline
1985 & 50\% & 19.2\% & 18-55\% & \$596,848 & \$1,119,090 & 100\% & \$14,921 \\
\hline
1989 & 28\% & 20.4\% & 18-55\% & \$784,896 & \$981,120 & 100\% & \$13,081 \\
\hline
1993 & 39.6\% & 21.9\% & 18-55\% & \$686,784 & \$858,480 & 100\% & \$11,446 \\
\hline
1995 & 39.6\% & 23.8\% & 18-55\% & \$649,992 & \$812,490 & 100\% & \$10,833 \\
\hline
1998 & 39.6\% & NA & 18-55\% & \$638,750 & \$766,500 ${ }^{4}$ & 100\% & \$10,220 ${ }^{5}$ \\
\hline
\end{tabular}
\end{center}
\end{table}

All dollar figures converted to 1999 dollars using the CPI All Urban, All Items Research Series. The adjustment factors used are $4.40,2.42,1.91,1.48,1.30,1.14,1.08$, and 1.02 for 1963, 1977, 1980, 1985, 1989, 1993, 1995, and 1998.\\
${ }^{1}$ The effective tax rate is the effective individual income tax rate. This is calculated by dividing individual income tax by total income.\\
${ }^{2}$ The annual exclusion is per donee.\\
${ }^{3}$ Since 1977 the gift tax range has been the same as the estate tax range. Prior to 1977 the gift tax range was $2.25-57.75 \%$.\\
${ }^{4}$ Starting in 1998 the estate tax exemption increases yearly to 1 million dollars in 2006 and the exclusion for closely held business is indexed for inflation.\\
${ }^{5}$ Starting in 1998 the annual exclusion is indexed for inflation.

\section*{Sources:}
For marginal and effective rates prior to 1980 , see Brownlee (2000). For marginal rates from 1980-1998, see Booth (1998) in the references. For effective rates from 1980-93, see Slemrod (1994). For effective rates for 1995, see Kasten, Sammartino, and Weiner (1998). For estate and gift tax information, see Johnson and Eller (1998) and Joulfaian (1998).

\begin{table}[h]
\begin{center}
\captionsetup{labelformat=empty}
\caption{Table 2: Ownership Rates of Assets and Liabilities}
\begin{tabular}{|l|l|l|l|l|l|l|l|}
\hline
\multirow{3}{*}{} & \multicolumn{5}{|c|}{\multirow{2}{*}{}} & \multicolumn{2}{|c|}{Averages} \\
\hline
 &  &  &  &  &  & Top 1\% 1962-95 & 0-99\% 1962-95 \\
\hline
 & 1962 & 1983 & 1989 & 1992 & 1995 &  &  \\
\hline
Financial Assets & 100.0 & 100.0 & 100.0 & 100.0 & 100.0 & 100.0 & 91.7 \\
\hline
Transaction and savings accts & 91.6 & 99.6 & 100.0 & 100.0 & 99.9 & 98.2 & 84.6 \\
\hline
Certificates of deposit & na & 38.7 & 32.7 & 31.5 & 24.6 & 31.9 & 14.0 \\
\hline
US Savings bonds & 36.7 & 23.2 & 18.1 & 18.9 & 28.0 & 25.0 & 23.3 \\
\hline
Federal, state and local bonds & 30.7 & 45.1 & 41.2 & 38.3 & 29.9 & 37.0 & 2.4 \\
\hline
Other bonds & 30.3 & 9.6 & 16.2 & 21.2 & 12.8 & 18.0 & 1.7 \\
\hline
Stocks & 84.0 & 79.9 & 72.5 & 69.7 & 65.0 & 74.2 & 16.3 \\
\hline
Mutual funds & 24.0 & 33.8 & 39.1 & 46.0 & 45.0 & 37.6 & 7.5 \\
\hline
Defined contribution pensions & 10.1 & 65.6 & 71.3 & 76.1 & 78.6 & 60.4 & 28.7 \\
\hline
Defined benefit pensions & na & 24.3 & 13.7 & 20.1 & 10.6 & 17.2 & 20.3 \\
\hline
Cash value of life insurance & 59.6 & 71.6 & 64.5 & 57.8 & 60.0 & 62.7 & 37.0 \\
\hline
Other managed assets & 13.4 & 24.9 & 27.5 & 17.4 & 17.7 & 20.2 & 3.4 \\
\hline
Other financial assets & 89.3 & 10.1 & 31.9 & 31.1 & 25.2 & 37.5 & 27.2 \\
\hline
Non-financial Assets & 96.7 & 100.0 & 99.3 & 100.0 & 99.8 & 99.2 & 89.3 \\
\hline
Primary residence & 74.1 & 96.6 & 86.0 & 93.5 & 96.0 & 89.2 & 62.2 \\
\hline
Vehicles & 77.7 & 91.4 & 90.6 & 97.8 & 89.5 & 89.4 & 82.5 \\
\hline
Investment Real Estate & 44.0 & 74.2 & 76.6 & 75.4 & 61.3 & 66.3 & 16.3 \\
\hline
Privately held businesses & 69.0 & 88.0 & 73.4 & 69.7 & 74.3 & 74.9 & 12.8 \\
\hline
Other non-financial assets & 50.8 & 30.7 & 57.1 & 54.6 & 46.1 & 47.9 & 11.2 \\
\hline
Debt & 50.2 & 77.9 & 77.3 & 80.3 & 70.5 & 71.2 & 71.5 \\
\hline
Mortgage & 30.7 & 54.5 & 35.8 & 53.4 & 52.5 & 45.4 & 37.6 \\
\hline
Other real estate debt & 12.8 & 45.2 & 48.3 & 53.6 & 35.1 & 39.0 & 6.6 \\
\hline
Student loans & na & na & 0.7 & 0.1 & 0.9 & 0.6 & 6.3 \\
\hline
Other installment loans & 20.7 & 35.0 & 19.5 & 21.9 & 12.6 & 21.9 & 46.9 \\
\hline
Credit cards & na & 9.6 & 14.8 & 17.0 & 12.2 & 13.4 & 33.8 \\
\hline
Other debt & 17.6 & 15.2 & 20.2 & 30.8 & 17.2 & 20.2 & 9.4 \\
\hline
\end{tabular}
\end{center}
\end{table}

Source: Survey of Financial Characteristics of Consumers and Surveys of Cosumer Finances\\
Note: Cells with an "na" indicate asset or debt categories not disaggregated in a particular survey year

\section*{Definitions of assets and debts}
Transaction and savings accounts include checking, saving, money market, and call accounts.\\
Federal, state, and local bonds include government bonds (not US Savings bonds) and municipal bonds.\\
Other bonds include mortgage, corporate, foreign, and other types of bonds.\\
Defined contribution pensions include employer-sponsored plans and personal retirement accounts.\\
Cash value of life insurance refers to the cash value of whole life policies.\\
Other managed assets consists of trusts, annuities and managed investment accounts.\\
Other financial assets consist royalties, future proceeds from lawsuits, oil, gas, and mineral leases, etc.\\
Other non-financial assets include such items as artwork, jewelry, etc.\\
Businesses include those in which the household has an active and/or passive interest.\\
Mortgage debt includes any borrowing on home equity lines of credit.\\
Installment loans consists of vehicle loans, home improvement loans (not home equity loans), and other loans.\\
Other debt includes other lines of credit, loans against pensions, loans against life insurance policies, margin loans, etc.

\begin{table}[h]
\begin{center}
\captionsetup{labelformat=empty}
\caption{Table 3: Composition of Net Worth}
\begin{tabular}{|l|l|l|l|l|l|l|l|}
\hline
\multirow{3}{*}{} & \multicolumn{5}{|c|}{\multirow[b]{2}{*}{Top 1 Percent of Households By Net Worth}} & \multicolumn{2}{|c|}{Averages} \\
\hline
 &  &  &  &  &  & Top 1\% 1962-95 & 0-99\% 1962-95 \\
\hline
 & 1962 & 1983 & 1989 & 1992 & 1995 &  &  \\
\hline
Financial Assets/Net Worth & 57.4 & 36.6 & 32.0 & 32.0 & 40.8 & 39.7 & 36.4 \\
\hline
As a fraction of total financial assets &  &  &  &  &  &  &  \\
\hline
Transaction and savings accts & 6.5 & 7.6 & 18.4 & 14.6 & 14.6 & 12.3 & 22.4 \\
\hline
Certificates of deposit & na & 2.3 & 3.9 & 2.5 & 2.7 & 2.9 & 9.4 \\
\hline
US Savings bonds & 1.0 & 0.3 & 0.4 & 0.4 & 0.2 & 0.4 & 2.8 \\
\hline
Federal, state and local bonds & 7.3 & 12.4 & 12.8 & 13.4 & 11.5 & 11.5 & 3.7 \\
\hline
Other bonds & 1.2 & 0.5 & 4.0 & 3.1 & 2.1 & 2.2 & 1.0 \\
\hline
Stocks & 53.6 & 39.9 & 23.2 & 30.8 & 26.5 & 34.8 & 15.2 \\
\hline
Mutual funds & 3.4 & 3.1 & 7.0 & 7.6 & 15.7 & 7.4 & 6.6 \\
\hline
Defined contribution pensions & 0.6 & 5.8 & 9.2 & 13.0 & 12.7 & 8.3 & 19.4 \\
\hline
Cash value of life insurance & 4.9 & 3.0 & 3.5 & 1.7 & 3.7 & 3.4 & 11.8 \\
\hline
Other managed assets & 21.5 & 24.5 & 12.6 & 8.0 & 7.9 & 14.9 & 4.8 \\
\hline
Other financial assets & 0.0 & 0.7 & 5.1 & 4.9 & 2.3 & 2.6 & 2.8 \\
\hline
Nonfinancial Assets/Net Worth & 46.6 & 69.4 & 76.8 & 74.9 & 64.3 & 66.4 & 86.7 \\
\hline
As a fraction of Net Worth &  &  &  &  &  &  &  \\
\hline
Primary residence & 5.2 & 7.8 & 7.8 & 8.9 & 7.1 & 7.4 & 49.6 \\
\hline
Vehicles & 0.4 & 0.3 & 0.8 & 0.7 & 0.7 & 0.6 & 6.3 \\
\hline
Investment real estate & 7.2 & 16.8 & 25.6 & 23.0 & 11.9 & 16.9 & 13.1 \\
\hline
Net value of private businesses & 30.6 & 39.3 & 38.5 & 39.0 & 41.4 & 37.7 & 14.8 \\
\hline
Other non-financial assets & 3.2 & 5.2 & 4.0 & 3.3 & 3.3 & 3.8 & 2.9 \\
\hline
Debt/Net Worth & 4.0 & 6.0 & 8.7 & 6.9 & 5.1 & 6.1 & 23.1 \\
\hline
As a Fraction of Net Worth &  &  &  &  &  &  &  \\
\hline
Mortgage & 0.6 & 0.8 & 0.9 & 1.6 & 1.6 & 1.1 & 15.5 \\
\hline
Other real estate debt & 1.5 & 3.1 & 6.7 & 4.4 & 2.6 & 3.7 & 3.2 \\
\hline
Student loans & na & na & 0.0 & 0.0 & 0.0 & 0.0 & 0.3 \\
\hline
Other installment loans & 0.7 & 0.8 & 0.3 & 0.2 & 0.1 & 0.4 & 3.1 \\
\hline
Credit cards & na & 0.0 & 0.0 & 0.0 & 0.0 & 0.0 & 0.6 \\
\hline
Other debt & 1.2 & 1.3 & 0.8 & 0.7 & 0.7 & 0.9 & 0.5 \\
\hline
Memo items &  &  &  &  &  &  &  \\
\hline
Median net worth (th '98 \$) & 1,841 & 4,291 & 4,720 & 4,138 & 4,748 & 3,948 & 51 \\
\hline
Avg net worth (th '98 \$) & 3,044 & 7,156 & 7,185 & 6,399 & 7,854 & 6,328 & 133 \\
\hline
Median wealth to income ratio & 15.4 & 14.9 & 18.8 & 20.8 & 20.0 & 18.0 & 1.5 \\
\hline
Avg wealth to income ratio & 18.5 & 27.4 & 33.0 & 35.7 & 38.6 & 30.6 & 8.7 \\
\hline
\end{tabular}
\end{center}
\end{table}

Source: Survey of Financial Characteristics of Consumers and Surveys of Consumer Finances\\
Note: Cells with an "na" indicate asset or debt categories not disaggregated in a particular survey year

\begin{table}[h]
\begin{center}
\captionsetup{labelformat=empty}
\caption{Table 4: Composition of Net Worth by Risk Category}
\begin{tabular}{|l|l|l|l|l|l|l|l|}
\hline
\multirow{3}{*}{} & \multicolumn{5}{|c|}{} & \multicolumn{2}{|c|}{Averages} \\
\hline
 & \multicolumn{5}{|c|}{Top 1 Percent of Households By Net Worth} & Top 1\% & 0-99\% \\
\hline
 & 1962 & 1983 & 1989 & 1992 & 1995 & 1962-95 & 1962-95 \\
\hline
Financial Assets/Net Worth & 57.4 & 36.6 & 32.0 & 32.0 & 40.8 & 39.7 & 36.4 \\
\hline
 &  & \multicolumn{4}{|l|}{As a fraction of Financial Assets} &  &  \\
\hline
Safe & 17.9 & 30.7 & 47.6 & 43.9 & 44.7 & 37.0 & 64.1 \\
\hline
Clearly safe & 7.5 & 10.2 & 22.6 & 17.5 & 17.5 & 15.1 & 34.7 \\
\hline
Fairly safe & 10.4 & 20.5 & 25.0 & 26.5 & 27.2 & 21.9 & 29.5 \\
\hline
Risky & 82.1 & 69.3 & 52.4 & 56.1 & 55.3 & 63.0 & 35.9 \\
\hline
 & \multicolumn{5}{|c|}{As a fraction of Net Worth} &  &  \\
\hline
Nonfinancial Assets & 46.6 & 69.4 & 76.8 & 74.9 & 64.3 & 66.4 & 86.7 \\
\hline
Primary residence & 5.2 & 7.8 & 7.8 & 8.9 & 7.1 & 7.4 & 49.6 \\
\hline
Investment real estate & 7.2 & 16.8 & 25.6 & 23.0 & 11.9 & 16.9 & 13.1 \\
\hline
Business equity & 30.6 & 39.3 & 38.5 & 39.0 & 41.4 & 37.7 & 14.8 \\
\hline
Other non-financial assets & 3.6 & 5.5 & 4.8 & 4.0 & 4.0 & 4.4 & 9.2 \\
\hline
Debt & 4.0 & 6.0 & 8.7 & 6.9 & 5.1 & 6.1 & 23.1 \\
\hline
Mortgage & 0.6 & 0.8 & 0.9 & 1.6 & 1.6 & 1.1 & 15.5 \\
\hline
Other Secured & 2.3 & 3.2 & 6.9 & 4.7 & 2.9 & 4.0 & 8.9 \\
\hline
Unsecured & 0.7 & 2.1 & 0.9 & 0.6 & 0.5 & 1.0 & 2.5 \\
\hline
Memo: & \multicolumn{5}{|c|}{} &  &  \\
\hline
Risky assets - narrow & 30.8 & 14.6 & 7.4 & 9.8 & 10.8 & 14.7 & 5.4 \\
\hline
Risky assets - broad & 84.9 & 81.4 & 80.8 & 80.0 & 75.8 & 80.6 & 41.2 \\
\hline
Risky assets - broadest & 90.9 & 88.9 & 88.8 & 88.4 & 86.9 & 88.8 & 52.0 \\
\hline
Mortgage debt / total debt & 14.1 & 12.8 & 10.6 & 22.6 & 31.8 & 18.4 & 66.6 \\
\hline
\end{tabular}
\end{center}
\end{table}

\section*{Definitions of asset and debt classifications:}
Clearly safe includes transaction accounts (checking, saving, money market, and call accounts), certificates of deposit, and US Savings bonds\\
Fairly safe includes state/local bonds, the fairly safe component of mutual funds' the fairly safe component of defined contribution pensions ${ }^{\dagger}$, and the cash value of life insurance policies.\\
Risky includes stocks, bonds (all types but state/local and US Savings), other managed assets, other financial assets, and the risky component of mutual funds and defined contribution pension accounts.

Risky assets - narrow consists of direct stock holdings\\
Risky assets - broad consists of risky financial assets, plus the net value of businesses and investment real estate\\
Risky assets - broadest consists of all assets in the broad definition and probably safe assets\\
Secured debt includes vehicle loans, loans against pensions and life insurance policies, investment real estate debt, and call account debt.\\
Unsecured debt includes credit card balances, installment loans, other lines of credit, and other debt.\\
†Definitions of the risky and 'fairly safe' components of mutual funds and defined contribution pensions follow below.

Sources: Calculations by the author using the Survey of Financial Characteristics of Consumers and Surveys of Consumer Finances

Calculations of fairly risky and fairly safe mutual funds and defined contribution pensions in Table 4 are as follows.

1962 SFCC Due to the lack of information on mutual fund investment strategies, all mutual funds are classified as risky and all defined contribution pensions are classified as safe.

1983 SCF The 1983 SCF did not ask about the investment strategy or risk characteristics of mutual funds or retirement accounts, so we had to make educated guesses based on other information. Tax-free mutual funds were allotted to the 'fairly safe' category because such funds consist almost exclusively of state and local government bonds, direct holdings of which we put in this category. Taxable mutual funds were allotted to the 'risky' category, because in the early 1980s these funds typically contained a mix of stocks and bonds. The calculation of risky and fairly safe, and clearly safe defined contribution pensions uses the institution that held the IRA/Keogh accounts as a proxy for investment direction. If a real estate investment company held the accounts, then those defined contribution pensions were considered risky. If a commercial bank, savings and loan, or credit union held the accounts, then those assets were considered fairly safe. If a brokerage, insurance company, employer, school/college/university, investment management company, or the AARP held the accounts, the defined contribution pensions were split $50 / 50$ between the fairly safe and risky. In the case that the household had no IRA/Keogh accounts, but had a thrift pension account, the assets were considered fairly safe.

1989-1995 SCF These surveys asked about the investment strategy for mutual funds and retirement accounts. Funds and accounts that consisted exclusively of one category of asset (such as stock or bond mutual funds) we allocated in the same way that we allocated direct holdings of that asset type. Mutual funds and accounts that contained a mix of stocks and bonds were allocated half-and-half to the 'fairly safe' and 'risky' categories. Accounts invested in real estate, commodities or limited partnerships were put in the 'risky' category.

\begin{table}[h]
\begin{center}
\captionsetup{labelformat=empty}
\caption{Table 5: Degree of Diversification of Portfolio Structure}
\begin{tabular}{|l|l|l|l|l|l|l|l|}
\hline
\multicolumn{3}{|l|}{Asset Combinations} & 1962 & 1983 & 1989 & 1992 & 1995 \\
\hline
\multicolumn{8}{|l|}{\begin{tabular}{l}
CS \\
FS \\
R \\
\end{tabular}} \\
\hline
0 & 0 & 0 & 0.0 & 0.0 & 0.0 & 0.0 & 0.0 \\
\hline
0 & 0 & 1 & 0.0 & 0.0 & 0.0 & 0.0 & 0.0 \\
\hline
0 & 1 & 0 & 1.1 & 0.0 & 0.0 & 0.0 & 0.1 \\
\hline
0 & 1 & 1 & 1.8 & 0.0 & 0.0 & 0.0 & 0.0 \\
\hline
1 & 0 & 0 & 0.0 & 2.5 & 1.6 & 6.5 & 2.4 \\
\hline
1 & 0 & 1 & 4.4 & 1.6 & 10.8 & 5.1 & 9.4 \\
\hline
1 & 1 & 0 & 10.9 & 12.8 & 9.1 & 9.4 & 10.7 \\
\hline
1 & 1 & 1 & 81.8 & 83.2 & 78.5 & 78.9 & 77.4 \\
\hline
\end{tabular}
\end{center}
\end{table}

Notes:\\
CS denotes clearly safe financial assets\\
FS denotes fairly safe financial assets\\
R denotes risky financial assets\\
0 denotes no ownership of assets in the specified category; 1 denotes ownership.\\
Note: A description of the asset classifications appears in the notes at the end of Table 4.\\
Source: Author's calculations using the Survey of Financial Characteristics of Consumers and Surveys of Consumer Finances

\begin{table}[h]
\begin{center}
\captionsetup{labelformat=empty}
\caption{Table 6: Risk Bearing By Age}
\begin{tabular}{|l|l|l|l|l|l|l|}
\hline
\multirow[b]{2}{*}{Survey} & \multicolumn{6}{|c|}{Age of Household Head} \\
\hline
 & <30 & 30-39 & 40-49 & 50-59 & 60-69 & >70 \\
\hline
\multicolumn{7}{|c|}{} \\
\hline
\multicolumn{7}{|l|}{1962 SFCC} \\
\hline
Risky fin asset ownership & 81.4 & 93.0 & 90.7 & 71.4 & 79.8 & 97.9 \\
\hline
Risky fin asset / fin assets & 85.5 & 96.7 & 73.7 & 83.1 & 75.1 & 77.5 \\
\hline
Broad risky asset ownership & 100.0 & 100.0 & 100.0 & 100.0 & 100.0 & 100.0 \\
\hline
Broad risky asset / total assets & 78.8 & 92.2 & 80.4 & 84.2 & 76.9 & 74.9 \\
\hline
\multicolumn{7}{|l|}{1983 SCF} \\
\hline
Risky fin asset ownership & 47.4 & 72.6 & 65.3 & 96.9 & 100.0 & 84.0 \\
\hline
Risky fin asset / fin assets & 74.2 & 57.6 & 86.9 & 78.3 & 68.0 & 65.8 \\
\hline
Broad risky asset ownership & 100.0 & 100.0 & 100.0 & 100.0 & 100.0 & 100.0 \\
\hline
Broad risky asset / total assets & 67.8 & 74.9 & 87.8 & 85.2 & 72.1 & 74.8 \\
\hline
\multicolumn{7}{|l|}{1989 SCF} \\
\hline
Risky fin asset ownership & 60.1 & 73.8 & 89.1 & 89.9 & 92.3 & 95.2 \\
\hline
Risky fin asset / fin assets & 92.5 & 47.3 & 61.6 & 51.2 & 51.0 & 51.8 \\
\hline
Broad risky asset ownership & 100.0 & 100.0 & 95.5 & 100.0 & 100.0 & 99.3 \\
\hline
Broad risky asset / total assets & 91.8 & 58.7 & 78.8 & 80.1 & 80.2 & 70.9 \\
\hline
\multicolumn{7}{|l|}{1992 SCF} \\
\hline
Risky fin asset ownership & 64.1 & 89.0 & 77.2 & 89.5 & 88.1 & 84.3 \\
\hline
Risky fin asset / fin assets & 65.1 & 52.9 & 48.4 & 61.0 & 58.8 & 58.8 \\
\hline
Broad risky asset ownership & 92.0 & 100.0 & 100.0 & 100.0 & 100.0 & 100.0 \\
\hline
Broad risky asset / total assets & 76.4 & 72.7 & 79.0 & 75.6 & 78.9 & 73.6 \\
\hline
\multicolumn{7}{|l|}{1995 SCF} \\
\hline
Risky fin asset ownership & 88.2 & 73.6 & 87.3 & 75.5 & 87.2 & 89.7 \\
\hline
Risky fin asset / fin assets & 27.2 & 60.1 & 59.1 & 50.7 & 51.2 & 57.9 \\
\hline
Broad risky asset ownership & 91.3 & 97.5 & 100.0 & 100.0 & 100.0 & 100.0 \\
\hline
Broad risky asset / total assets & 41.8 & 67.5 & 78.8 & 74.2 & 68.4 & 70.7 \\
\hline
\multicolumn{7}{|l|}{1962-1995} \\
\hline
Risky fin asset ownership & 68.2 & 80.4 & 81.9 & 84.6 & 89.5 & 90.2 \\
\hline
Risky fin asset / fin assets & 68.9 & 62.9 & 65.9 & 64.9 & 60.8 & 62.3 \\
\hline
Broad risky asset ownership & 96.7 & 99.5 & 99.1 & 100.0 & 100.0 & 99.9 \\
\hline
Broad risky asset / total assets & 71.3 & 73.2 & 81.0 & 79.9 & 75.3 & 73.0 \\
\hline
\end{tabular}
\end{center}
\end{table}

Notes: The definition of risky financial assets corresponds to the sum of clearly risky and fairly risky assets defined in Table 4. The definition of broad risky assets corresponds to the 'risky assets - broad' classification in Table 4.\\
Source: Survey of Financial Characteristics of Consumers and Surveys of Consumer Finances

\begin{table}[h]
\begin{center}
\captionsetup{labelformat=empty}
\caption{Table 7: International Comparisons}
\begin{tabular}{|l|l|l|l|l|l|l|l|l|l|l|}
\hline
\multirow{2}{*}{} & \multicolumn{2}{|c|}{US - 1995} & \multicolumn{2}{|c|}{Netherlands - 1995} & \multicolumn{2}{|c|}{Italy - 1995} & \multicolumn{2}{|c|}{Germany - 1993} & \multicolumn{2}{|c|}{UK - 1997/98} \\
\hline
 & Top 5\% & Bot 95\% & Top 5\% & Bot 95\% & Top 5\% & Bot 95\% & Top 5\% & Bot 95\% & Top 5\% & Bot 95\% \\
\hline
Gross Financial Assets (GFA) per HH & \$1,120,583 & \$41,118 & 278,778 & 21,138 & 122,507 & 17,286 & 155,623 & 28,022 & 175,427 & 10,720 \\
\hline
As a ratio to Gross Financial Assets & 100.0 & 100.0 & 100.0 & 100.0 & 100.0 & 100.0 & 100.0 & 100.0 & 100.0 & 100.0 \\
\hline
Safe & 37.0 & 54.1 & 21.8 & 47.8 & 52.7 & 81.2 & 68.3 & 79.5 & 54.5 & 77.2 \\
\hline
Clearly safe & 17.9 & 27.6 & 14.8 & 45.2 & 46.2 & 69.7 & 11.3 & 27.0 & 46.3 & 70.8 \\
\hline
Fairly safe & 19.2 & 26.5 & 7.0 & 2.6 & 6.5 & 11.5 & 57.0 & 52.5 & 8.2 & 6.4 \\
\hline
Risky & 50.7 & 29.9 & 52.9 & 14.3 & 30.6 & 9.2 & 19.7 & 10.0 & 25.1 & 15.0 \\
\hline
Clearly risky & 44.6 & 27.4 & NA & NA & 13.9 & 2.1 & 12.0 & 4.0 & 17.1 & 12.9 \\
\hline
Fairly risky & 6.1 & 2.4 & NA & NA & 16.7 & 7.1 & 7.6 & 6.0 & 8.8 & 2.0 \\
\hline
Risk Characteristics Unknown & 12.3 & 16.0 & 25.3 & 37.9 & 16.7 & 9.6 & 12.0 & 10.4 & 19.6 & 7.8 \\
\hline
Nonfinancial Assets per HH & \$1,626,405 & \$98,423 & 432,098 & 34,539 & 905,204 & 112,588 & 661,115 & 96,306 & -- & -- \\
\hline
As a ratio to Net Worth & 63.1 & 95.5 & 69.4 & 116.0 & 88.1 & 86.7 & -- & -- & -- & -- \\
\hline
Gross value of primary residence & 12.7 & 65.7 & 31.9 & 98.5 & 29.2 & 53.7 & 88.0 & 86.0 & -- & -- \\
\hline
Net value of private business & 32.0 & 6.5 & 19.1 & 3.1 & 37.0 & 15.6 & -- & -- & -- & -- \\
\hline
Gross value investment real estate & 13.5 & 9.9 & 16.8 & 4.2 & 16.0 & 5.5 & -- & -- & -- & -- \\
\hline
Gross value of durables, of which & -- & -- & -- & -- & 3.7 & 10.1 & -- & -- & -- & -- \\
\hline
Vehicles & 1.3 & 10.8 & 1.7 & 10.1 & 2.6 & 7.3 & -- & -- & -- & -- \\
\hline
Other non-financial assets & 3.6 & 2.7 & NA & NA & 2.1 & 1.8 & -- & -- & -- & -- \\
\hline
Debt per HH & \$169,454 & \$36,479 & 75,240 & 16,197 & 27,319 & 4,753 & 65,382 & 15,111 & -- & -- \\
\hline
As a ratio to Net Worth & 6.6 & 35.4 & 14.1 & 51.5 & 2.7 & 3.7 & 8.7 & 16.4 & -- & -- \\
\hline
Mortgage on primary residence & 3.0 & 25.7 & 12.4 & 45.6 & 0.6 & 2.2 & -- & -- & -- & -- \\
\hline
Other secured debt & 3.1 & 5.9 & -- & -- & -- & -- & -- & -- & -- & -- \\
\hline
Unsecured debt & 0.5 & 3.8 & 1.7 & 5.4 & -- & -- & 0.4 & 1.8 & -- & -- \\
\hline
Net Worth per HH & \$2,577,534 & \$103,063 & 622,933 & 29,779 & 1,027,711 & 129,874 & 750,592 & 91,987 & -- & -- \\
\hline
Memo: &  &  &  &  &  &  &  &  &  &  \\
\hline
Clearly Risky Financial Assets/GFA & 44.6 & 27.4 & NA & NA & 13.9 & 2.1 & 12.0 & 4.0 & -- & -- \\
\hline
Broad Risky Assets/GFA & 155.3 & 71.0 & 20.8 & 3.4 & 475.8 & 167.6 & 19.7 & 10.0 & -- & -- \\
\hline
Very Broad Risky Assets/GFA & 174.5 & 97.5 & 52.2 & 8.2 & 482.3 & 179.1 & 76.7 & 62.5 & -- & -- \\
\hline
Total Household Income & \$216,142 & \$39,685 & 59,958 & 19,924 & 64,012 & 22,732 & 80,655 & 35,647 & -- & -- \\
\hline
Total Household Noncapital Income & \$135,864 & \$37,985 & NA & NA & 56,035 & 21,864 & 59,873 & 32,306 & -- & -- \\
\hline
\end{tabular}
\end{center}
\end{table}

\section*{Notes:}
\section*{Definitions Common to All Countries}
Households are sorted once, by the broadest measure of net worth available, to determine their classification into top 5 or bottom 95 percent. Asset shares are computed as ratio of averages. All statistics use sample weights.

\section*{Definitions for United States}
Data are drawn from the 1995 Survey of Consumer Finances. Figures are reported in 1999 dollars, converted from 1995 numbers using the CPI-U-RS.

\section*{Definitions of financial asset classifications:}
Clearly safe includes transaction accounts (checking, saving, money market), certificates of deposit, US Savings bonds, and mutual funds invested exclusively in these assets Fairly safe includes state/local bonds, mutual funds, and other managed assets invested in state/local bonds, and the cash value of life insurance policies.\\
Fairly risky includes bonds (all types except state/local and US Savings) and mutual funds invested in bonds (all types except state/local and US Savings) Clearly risky includes stocks and financial assets invested in real estate, commodities, and private partnerships\\
Risk characteristics unknown. The three largest components of this category are mutual funds whose investment direction is unknown, retirement accounts whose investment direction is unknown, and other managed assets whose investment direction is unknown. 'Other financial assets' are also included in this category.\\
Mutual funds and retirement accounts which are invested in a single category of assets (for example, 100 percent stock mutual funds) are included in the corresponding category\\
Other secured debt includes call account debt, vehicle loans, loans against pensions and life insurance policies and loans for investment real estate\\
Unsecured debt includes credit card balances, installment loans, other lines of credit, and other misc. debts\\
Broad Risky Assets - consists of clearly risky and fairly risky financial assets, plus businesses and investment real estate\\
Very Broad Risky Assets - consists of broad risky assets plus fairly safe assets\\
Total Household Income includes all income to the household from any source.\\
Total Noncapital Income subtracts all capital income (dividends, interest, capital gains, etc.) from Total Household Income

\section*{Definitions for the Netherlands}
Data are drawn from the 1995 CentER Savings Survey. For further information see the Netherlands country chapter.\\
1998 guilder are converted to Euros using the rate 1 Euro $=2.203$ guilders.\\
Clearly safe - transactions and savings accounts and certificates of deposit.\\
Clearly risky - Stocks, bonds, mutual funds\\
Risk characteristics unknown - The largest items are defined contribution pension plans, cash value of life insurance, and employer-sponsored pension plans.

\section*{Definitions for Italy}
All values are expressed in Euro, obtained by converting 1995 Lire to 1999 Lire using the increase in the CPI from 1995 to 1999 (10.8 percent) and converting to Euros using the 1999 fixed exchange rate between Euros and Lire, 1 Euro $=1.936$ Lire.\\
Data are drawn from the 1995 Survey of Household Income and Wealth, described in the Italy country chapter.\\
Clearly safe includes currency, transaction accounts (checking, saving, and postal accounts), certificates of deposit and short-term Treasury Bills.\\
Fairly safe includes the cash value of life insurance policies.\\
Fairly risky includes bonds (all types except short-term government bills), mutual funds and managed investment accounts.\\
Clearly risky includes only stocks.\\
Risk characteristics unknown: mutual funds and defined contribution pension funds.\\
Mortgage debt includes all mortgage debt, not just the primary residence.\\
Durables do not include art objects, jewelry, etc.\\
Broad Risky Assets - consists of clearly risky and fairly risky financial assets, plus businesses and investment real estate\\
Very Broad Risky Assets - consists of broad risky assets plus fairly safe assets

\section*{Definitions for Germany}
Data are drawn from the Income and Expenditure Survey wave 1993 covering 31,774 West German households and 8456 East German households, $80 \%$ subsample, excluding households with total net monthly income of $35,000 \mathrm{DM}(1993)$ or more. The data set is described in detail in the appendix to the German country study.\\
1993 DM are converted to 1999 DM using the CPI index 1 DM (December 1999) = 1.054878 DM (1993), and then to Euros by the fixed rate of 1.95583 DM = 1 Euro.\\
Fairly safe includes the cash value of endowment life insurance, assets accumulated in building society savings contracts (Bausparverträge), municipal bonds, savings certificates, and government bonds\\
Fairly risky includes other bonds and mutual funds invested in stocks or bonds\\
Clearly risky includes stocks and mutual funds invested in real estate\\
Risk characteristics unknown - "Other" financial assets.\\
Non-financial assets - no data are available separating real estate into personal residence and other, so the number for personal residence reflects all real estate

\section*{Definitions for UK}
Data are drawn from the 1997-98 Financial Research Survey - see UK country chapter for details.\\
Figures were calculated in 1997 pounds, converted to 1999 pounds using the CPI inflation factor of 1.0726, then converted to Euros using the 1999 Euro/pound exchange rate of 1.7.\\
Clearly safe includes transaction accounts (checking, saving, money market), certificates of deposit, National Savings current accounts, Premium bonds and TESSAs.\\
Fairly safe includes government and local bonds, plus National Savings Bonds.\\
Fairly risky includes all bonds (all types except government/local and National savings)\\
Clearly risky stocks and shares\\
Risk characteristics unknown. This is almost entirely mutual funds where investment direction is unknown and retirement accounts whose investment direction is unknown; 'Other financial assets' are also included in this category.\\
Other secured debt includes call account debt, vehicle loans, loans against pensions and life insurance policies and loans for investment real estate\\
Secured and unsecured debt includes installment loans, other lines of credit, and other misc. debts, agreed overdrafts and vehicle or other secured loans\\
Value of pension and life insurance assets not known in survey and hence not included in definition of gross financial assets, or in any subcomponent.

\begin{table}[h]
\begin{center}
\captionsetup{labelformat=empty}
\caption{Table 8: Risk Aversion By Income and Net Worth, 1992 and 95 SCFs}
\begin{tabular}{|l|l|l|l|l|}
\hline
Survey Year & \multicolumn{2}{|c|}{1992} & \multicolumn{2}{|c|}{1995} \\
\hline
 & Mean & \% No Risk & Mean & \% No Risk \\
\hline
Permanent Income Percentiles &  &  &  &  \\
\hline
99-100 & 2.5 & 3.8 & 2.6 & 6.2 \\
\hline
80-98.9 & 2.8 & 16.9 & 2.8 & 16.1 \\
\hline
0-79.9 & 3.3 & 48.7 & 3.2 & 40.1 \\
\hline
Net Worth Percentiles &  &  &  &  \\
\hline
99-100 & 2.6 & 11.5 & 2.5 & 6.5 \\
\hline
80-98.9 & 2.9 & 21.6 & 2.8 & 17.6 \\
\hline
0-79.9 & 3.3 & 48.4 & 3.2 & 43.8 \\
\hline
\end{tabular}
\end{center}
\end{table}

Notes: The table summarizes answers to the following question: "Which of the statements on this page comes closest to the amount of financial risk that you (and your spouse/partner) are willing to take when you save or make investments? 1. Take sustantial financial risks expecting to earn substantial returns; 2. Take above average financial risks expecting to earn above average returns; 3. Take average financial risks expecting to earn average returns; 4. Not willing to take any financial risks. To tabulate results by 'permanent income' percentile, the sample was restricted to those households who said that their income in the survey year was 'about normal,' and permanent income was defined as observed income for such households.

\begin{table}[h]
\begin{center}
\captionsetup{labelformat=empty}
\caption{Table 9: Effect of Inheritances on the Change in Risk Aversion}
\begin{tabular}{|l|l|l|}
\hline
\multirow{2}{*}{} & \multicolumn{2}{|c|}{Sample} \\
\hline
 & All Recipients & Surprised \\
\hline
LINH & -0.186 *** (0.027) & -0.199 *** (0.032) \\
\hline
MARRIED & $0.220^{* * *}$ (0.077) & $0.310^{* * *}$ (0.103) \\
\hline
KIDS & -0.005 (0.036) & 0.002 (0.064) \\
\hline
A2 & 0.116 (0.099) & 0.119 (0.134) \\
\hline
A3 & 0.394 *** (0.114) & 0.495 *** (0.147) \\
\hline
UNEMP83 & -0.214 * (0.116) & -0.422 *** (0.161) \\
\hline
UNEMP89 & -0.408 *** (0.108) & -0.468 *** 0.120 \\
\hline
OWNHOME & -0.149 (0.090) & -0.209 * (0.108) \\
\hline
EDUC & 0.014 (0.012) & 0.012 (0.014) \\
\hline
CONSTANT & 1.730 (0.290) & 1.848 (0.348) \\
\hline
Number of Obs & 0.132 & 0.159 \\
\hline
\end{tabular}
\end{center}
\end{table}

\footnotetext{Notes: Dependent variable DRISKAV is the change in attitude toward financial risk for the household between 1983 and 1989, as described in the text. A negative change implies a reduction in risk aversion. Standard errors in parentheses. *denotes significance at the 90 percent level; ** denotes significance at the 95 percent level; *** denotes significance at the 99 percent level. The first column ("All Recipients") reports results including all households who received an inheritance between 1983 and 1989. The second column ("Surprised") includes all households who received an inheritance between 1983 and 1989 but reported in 1983 that they did not expect ever to receive a substantial inheritance. The regression specification follows Gentry and Hubbard's baseline specification. Variable definitions:

LINH Log of total value of inheritances received between 1983 and 1989\\
MARRIED Dummy variable for hh head married in 1989\\
KIDS Number of kids under 18 in the household in 1983\\
A2 Age dummy, hh head between 35 and 54 in 1983\\
A3 Age dummy, hh head at least 55 in 1983\\
UNEMP83 Dummy variable for hh head unemployed in 1983 and employed in 1989\\
UNEMP89 Dummy variable for hh head employed in 1983 and unemployed in 1989\\
OWNHOME Dummy variable for hh being a homeowner in 1983\\
EDUC HH head years of education in 1989
}Percent of 1st, 2nd, and 3rd Largest Businesses in Total Business Equity and Aggregate Ratio of Business Derived Income to Total Income, for Households Owing Some Business Equity, by Net Worth Percentile, 1995 SCF

\begin{table}[h]
\begin{center}
\captionsetup{labelformat=empty}
\caption{Table 10: Lack of Diversification of Business Wealth}
\begin{tabular}{|l|l|l|l|}
\hline
\multirow{2}{*}{} & \multirow[b]{2}{*}{All HHs} & \multicolumn{2}{|c|}{Percentile} \\
\hline
 &  & 0-99\% & 99-100\% \\
\hline
Value of Business & \multicolumn{3}{|c|}{} \\
\hline
Largest & 0.88 & 0.89 & 0.82 \\
\hline
2 Largest & 0.92 & 0.92 & 0.92 \\
\hline
3 Largest & 0.92 & 0.92 & 0.94 \\
\hline
'Actively Managed' Business Assets & 0.89 & 0.89 & 0.85 \\
\hline
Ratio to Total Income: & \multicolumn{3}{|c|}{} \\
\hline
Business Income & 0.32 & 0.28 & 0.38 \\
\hline
Business Derived Income & 0.58 & 0.66 & 0.44 \\
\hline
\end{tabular}
\end{center}
\end{table}

Notes:\\
Value in cells in the first four rows is the mean ratio of the value of the business(es) to total business equity.\\
Business income is the income the household reported receiving from all the businesses the household owns, regardless of whether anyone in the household works in any of the businesses.\\
Business derived income consists of wages and salaries paid by the business to the household head or spouse plus retained earnings reported by the head or spouse from a business the household owns.

Source: Calculations by author using the 1995 Survey of Consumer Finances

\begin{table}[h]
\begin{center}
\captionsetup{labelformat=empty}
\caption{Table 11: Riskiness of Financial Assets in Entrepreneurs' Portfolios}
\begin{tabular}{|l|l|l|l|l|l|l|}
\hline
\multirow{3}{*}{} & \multicolumn{6}{|c|}{Average Portfolio Allocations for Business Owners and Nonowners, 1995 SCF} \\
\hline
 & \multicolumn{3}{|c|}{Top 1\%} & \multicolumn{3}{|c|}{0-99\%} \\
\hline
 & Bus $=0$ & 0<Bus<33 & 33<Bus<100 & Bus=0 & 0<Bus<33 & 33<Bus<100 \\
\hline
Financial Assets/Net Worth & 77.3 & 59.0 & 16.9 & 41.5 & 36.0 & 18.2 \\
\hline
 & \multicolumn{3}{|c|}{As a Percent of Financial Assets} & \multicolumn{3}{|c|}{As a Percent of Financial Assets} \\
\hline
Safe & 37.0 & 55.3 & 58.4 & 76.1 & 68.2 & 81.1 \\
\hline
Clearly safe & 13.0 & 18.8 & 26.7 & 47.5 & 37.0 & 55.3 \\
\hline
Fairly safe & 24.1 & 36.5 & 31.7 & 28.6 & 31.2 & 25.8 \\
\hline
Risky & 63.0 & 44.7 & 41.5 & 24.0 & 31.8 & 18.9 \\
\hline
\end{tabular}
\end{center}
\end{table}

Notes:\\
Only households with $\geq \$ 1000$ in net worth are included in this table. BUS is defined as the ratio of noncorporate business equity to total net worth. Definitions of financial assets, safe, and risky assets are as in previous tables.\\
Source: Calculations by the author using the 1995 Survey of Consumer Finances

\begin{figure}[h]
\begin{center}
  \includegraphics[alt={},max width=\textwidth]{dbf54223-810e-4061-888e-286d1e76c342-42_757_1197_887_464}
\captionsetup{labelformat=empty}
\caption{Figure 1: Wealth Profiles for Baseline and More Patient Households Source: Reproduced from Carroll (2000)}
\end{center}
\end{figure}

\begin{figure}[h]
\begin{center}
  \includegraphics[alt={},max width=\textwidth]{dbf54223-810e-4061-888e-286d1e76c342-43_942_1499_797_313}
\captionsetup{labelformat=empty}
\caption{Figure 2: Age Profile of Log Wealth for the 99th Percentile, SCF Data Source: Reproduced from Carroll (2000)}
\end{center}
\end{figure}


\end{document}